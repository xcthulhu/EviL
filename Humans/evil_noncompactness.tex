In this section I shall demonstrate that \textsc{EviL} is not compact by giving an example of an infinite set of formulae for which every finite subset is satisfiable while the entirety is not.  %In the subsequent discussion I shall restrict to models with one agent and cease to employ subscripts.

\begin{lemma}%\renewcommand{\qedsymbol}{$\dashv$}
Let $\tau : \Phi \to \mathcal{L}$ be defined as follows:
\begin{center}
\begin{minipage}{3in}
\begin{tabbing}
	$\tau(p) := $ \=$p \wedge \Pos\top \wedge$ \\
	\> $\Nec$ \= $(\neg p \wedge \Pos\top \wedge$\\
	\> \> $\Nec$ \= $(p \wedge \Pos p))$
\end{tabbing}
\end{minipage}
\end{center}
Every finite subset of $\tau[\Phi]$ is satisfiable, but not the entirety, which is infinite. 
\end{lemma}
\begin{proof}
That $\tau[\Phi]$ is infinite is immediate, as $\Phi$ was stipulated to be infinite. 

So let $S\subseteq \tau[\Phi]$ by finite.  We shall find a model that satisfies $S$.  First observe that there is a finite $\Psi\subset \Phi$ such that $S = \tau[\Psi]$. Let $\{q,r,s\} \subseteq \Phi \; \bs \; \Psi$ be three distinct letters - this can be done, since $\Phi \; \bs \; \Psi$ is infinite.  Let $\mathfrak{M}$ be as in Fig. \ref{fig:mod4}, where 
\begin{align*}
a & := \Psi \cup \{s\} \\
b & := \{q \} \\
c & := \Psi \cup \{r\}
\end{align*}
%the valuation function $V$ of $\mathfrak{M}$\footnote{ is such that: $V(p) =\{a,c\}$ for each $p \in \Psi$, $V(q) = \{b\}$, $V(r) = \{c\}$, and $V(s) = \{a\}$.
One may check that $\mathfrak{M},( a,\{q\}) \VDash \tau(p)$ for all $p \in \Psi$.

	\begin{figure}[htbp] %  figure placement: here, top, bottom, or page
   \centering
	\begin{tikzpicture}[->,>=stealth',shorten >=1pt,auto,node distance=4cm,
                    semithick]
  \tikzstyle{every state}=[text=black, shape=rectangle]
  \tikzstyle{empt} = [draw=none,fill=none]
\begin{scope}
\node[empt] (aA1) at (45:2.5) [] {$\pr{a}{\{q\}}$};
\node[empt] (aA2) at (165:2.5) {$\pr{b}{\{r\}}$};
\node[empt] (aA3) at (285:2.5) {$\pr{c}{\{s\}}$};
%\node[empt] (aA4) at (270:2.5) {$\pr{w_4}{\{d\}}$};
\path
(aA1) edge [->,bend right] node {} (aA2)
(aA2) edge [->,bend right] node {} (aA3)
(aA3) edge [->,bend right] node {} (aA1);
%(aA3) edge [->,densely dotted,bend right] node {} (aA4)
%(aA4) edge [->,densely dotted,bend right] node {} (aA1);
\end{scope}
\end{tikzpicture}
\caption{Model $\mathfrak{M}$}
   \label{fig:mod4}
\end{figure}

On the other hand, suppose there was some model $\mathfrak{N}$ such
that $\mathfrak{N}, ( a, A) \VDash \tau(p)$ for all $p \in \Psi$.
This implies that $\mathfrak{N}, ( a, A) \VDash p$ for each $p \in
\Phi$.  Moreover, let $( b, B)$ be such that $\mathfrak{N},( b,
B)\VDash A$ (one exists since by hypothesis $\mathfrak{N}, ( a, A)
\VDash \Pos p$).  By the semantics of $\tau(p)$, it is evident that
$\mathfrak{N}, ( b, B) \VDash \neg p$ and that there's a $( c, C)$
such that $\mathfrak{N},( c,C ) \VDash B$ and $\mathfrak{N},( c,C )
\VDash p \wedge \Pos p$.  But then from this it must be that $a$ and
$c$ both contain exactly the same sentence letters, which means that
$\mathfrak{N},( a,A) \VDash B$ as well.  However it cannot be that
$\mathfrak{N},( a,A ) \VDash \Pos p$ since $\mathfrak{N},( a,A )
\VDash \Nec \neg p$ by $\tau(p)$. Thus it is impossible that
$\mathfrak{N},( a, A) \VDash \tau[\Phi]$. $\lightning$
\end{proof}

This failure of compactness, while a fairly basic result in the model
theory of \textsc{EviL}, is fundamental.  As a consequence, there is
no hope of achieving completeness using infinitary Lindenbaum
constructions that are typically employed in modal logic, such as is
done in \citep[][chapter 4]{blackburn_modal_2001}, for instance.
Any completeness theorem for \textsc{EviL} will necessarily have to be
finite in nature.