\subsection{Introduction}

In this final section, we discuss how \textsc{EviL} compares to
 two alternative logics of explicit
knowledge, the approach to epistemic logic suggested in 
\cite{van_benthem_reflectionsepistemic_1991}.
The two other approaches are:
\begin{bul}
\item \emph{Vel\'{a}zquez-Quesada Logic} ($VQL$) from \cite{velzquez-quesada_inference_2009}
\item \emph{Dynamic Awareness Logic} ($DAL$)  from \cite{van_benthem_inference_2009}
%\item \emph{Justification Logic} ($JL$), introduced in
%  \cite{fitting_logic_2005}
\end{bul}

Note that $VQL$ and $DAL$ are names we have introduced for the logics
discussed above, for ease of reference.

We will conclude with our final thoughts on \textsc{EviL}.

\subsection{Vel\'{a}zquez-Quesada Logic}

In this section we introduce Vel\'{a}zquez-Quesada Logic ($VQL$), the
logic introduced in \cite{velzquez-quesada_inference_2009}.  The
author originally called their logic \textbf{EI}, which stands for
``explicit/implicit.''  We have chosen the name $VQL$, as there is
more than one system that claims to model explicit and implicit
knowledge.
The idea behind the semantics of 
$VQL$ is to equip Kripke structures with special 
\emph{awareness} relations, as introduced in \cite{fagin_belief_1987}.
In \cite{velzquez-quesada_inference_2009}, the author notes himself
how his work compares to other awareness based approaches.  Notably,
his work is related to Ho Ngoc Duc's logic
\cite{duc_logical_1995,duc_reasoning_1997,duc_resource-bounded_2001},
Jago's logic for agents with bounded resources
\cite{jago_logics_2006}, and van Benthem's \emph{acts of
realization} \cite{van_benthem_merging_2008}.

$VQL$ Kripke structures have two awareness relations.  
The first relation establishes whether the agent is aware of
a particular fact, while the second establishes whether the agent is
aware of a rule.  By imposing various properties on these enriched 
Kripke structures, one can model inference via update modalities; this
will be illustrated shortly.

We note that early drafts of $VQL$ were presented employing 
multiple agents, while the version presented in
\cite{velzquez-quesada_inference_2009}
limited itself to the single agent case.

Formally, we can understand $VQL$ accordingly:
\begin{definition}
A \textbf{rule} $\rho \in \powerset_\omega \mathcal{L}_0(\Phi) \times
\mathcal{L}_0(\Phi)$ is a pair $(\Gamma, \gamma)$ where $\Gamma$ is a
finite set of propositional formulae and $\gamma$ is a single
propositional formula.

We shall abbreviate $\mathcal{R}(\Phi)$ as the set of rules over $\Phi$.
\end{definition}

\begin{definition}
The grammar $\mathcal{VQL}(\Phi)$ is defined as:
\[ \phi ::= p \in \Phi\ |\ \top\ |\ \phi \vee
\psi \ |\ \neg \phi \ |\ \Pos \phi  \ |\ I\gamma\ |\ L \rho \]
Here $\gamma \in \mathcal{L}_0(\Phi)$, while $\rho \in
\mathcal{R}(\Phi)$.  The other connectives are assumed to be 
defined as usual.
\end{definition}

\begin{definition}
$TR : \mathcal{R}(\Phi) \to \mathcal{L}_0(\Phi)$ is a function that
takes rules to first order formulae that codify them as follows:
\[ TR(\rho) := \bigwedge \Gamma \to \gamma \]
Where $\Gamma$ is the set of premises of $\rho$ and $\gamma$ is the conclusion.
\end{definition}

\begin{definition}
A $VQL$ Kripke structure is a model $\mathbb{V} = \langle W, R, V, Y,
Z\rangle$ where
\begin{bul}
  \item $W$ is a set of worlds
  \item $R \subseteq W \times W$ is an accessibility relation
  \item $V : W \to \powerset \Phi$ is valuation
  \item $Y : W \to \powerset \mathcal{L}_0(\Phi)$ is a propositional
    awareness function, relating worlds to sets of formulae the agent
    is aware of at that world
  \item $Z : W \to \mathcal{R}(\Phi)$ is a logical awareness function,
    relating worlds to sets of rules that the agent is aware of at
    those worlds
\end{bul}

Likewise, $VQL$ Kripke structures make true the following properties
(along with intuitive readings):
\begin{mynum}
\item $R$ is an equivalence relation
\item If $\gamma \in Y
  (w)$ and $w R v$  then $\gamma \in Y (v)$
\item If $\rho \in Z (w)$ 
  and $w R v$  then $\rho \in Z (v)$ -- once the agent is aware of
  some rule then that awareness persists
\item If $\gamma \in Y (w)$ then
  $\mathbbm{V}, w \Vdash_{VQL} \gamma$ -- the agent is only
  aware of facts
\item  If $\rho \in Z (w)$ then $\mathbbm{V}, w \Vdash_{VQL} \tmop{TR} (\rho)$
\end{mynum}
\end{definition}

We may intuitively read the above properties in the following manner:
\begin{mynum}
  \item \emph{Information States} -- $R$ partitions the universe into
    different information states, as one traditionally assumes in
    epistemic logic
  \item \emph{Assertoric Awareness Positive Introspection} -- Once the agent is
  aware of some assertion, they know that they are aware of that rule
    \item \emph{Logical Awareness Positive Introspection} -- Once the agent is
  aware of some rule, they know that they are aware of that rule
    \item \emph{Factiveness} -- If the agent is aware of proposition,
      then it is a fact
    \item \emph{Soundness} -- If the agent is aware of some rule, then
      that rule is sound
\end{mynum}

The following provides semantics for the grammar in terms of the
stipulated structures:
\begin{definition}
$VQL$'s truth predicate $\Vdash_{VQL}$ is defined such that:
\begin{align*}
  \mathbbm{V}, w \Vdash_{VQL} p & \Longleftrightarrow w \in V (p)\\
  \mathbbm{V}, w \Vdash_{VQL} \top & \tmop{always}\\
  \mathbbm{V}, w \Vdash_{VQL} \phi \vee \psi & \Longleftrightarrow
  \mathbbm{V}, w \Vdash_{VQL} \phi \textup{ \  or
  \  } \mathbbm{V}, w \Vdash_{VQL} \psi\\
  \mathbbm{V}, w \Vdash_{VQL} \neg \phi & \Longleftrightarrow
  \mathbbm{V}, w \nVdash_{VQL} \phi\\
  \mathbbm{V}, w \Vdash_{VQL} \Pos \phi & \Longleftrightarrow
  \exists v \in W.w R v\ \&\ \mathbbm{V}, v \Vdash_{VQL} \phi\\
  \mathbbm{V}, w \Vdash_{VQL} I \gamma & \Longleftrightarrow \gamma \in
  Y (w)\\
  \mathbbm{V}, w \Vdash_{VQL} L \rho & \Longleftrightarrow \rho \in Z
  (w)
\end{align*}
% \begin{equation*}
% \mathbb{M},w \Vdash_{VQL} p & \iff & w \in V(p) \\
% \mathbb{M},w \Vdash_{VQL} \top & \iff &  \\
% \mathbb{M},w \Vdash_{VQL} \phi \vee \psi & \iff & \mathbb{M},w
% \Vdash_{VQL} \phi \textup{ or }  \mathbb{M},w
% \Vdash_{VQL} \psi \\
% \mathbb{M},w \Vdash_{VQL} p & \iff & w \in V(p) \\
% \end{equation*}
\end{definition}

Table \ref{table:VQLaxioms} expresses the axioms of $VQL$. Certain
axioms impose certain structure on the models; soundness and
completeness theorems are discussed in
\cite{velzquez-quesada_inference_2009}. 

The
semantics for $VQL$ also supports modeling inference.  
The author chose to model inference using model updates, in the style of
dynamic epistemic logic as pioneered in \cite{gerbrandy_bisimulationsplanet_1998}.

\begin{table}
%\newcounter{rownum}
\setcounter{rownum}{0}
%\newcounter{rownum2}
\setcounter{rownum2}{0}
\begin{tabularx}{\linewidth}{|cl>{\it}X|}
\hline
& & \\
  (\refstepcounter{rownum}\arabic{rownum}) & All propositional
  tautologies & 
% \multirow{4}{8.5cm}{Axioms of classical
%     propositional logic} 
\\
%   (\refstepcounter{rownum}\arabic{rownum}) & $\vdash (\phi \rightarrow \psi \rightarrow \chi) \rightarrow (\phi
%   \rightarrow \psi) \rightarrow (\phi \rightarrow \chi)$ & \\
%   (\refstepcounter{rownum}\arabic{rownum}) & $\vdash (\neg \phi \rightarrow \neg \psi) \rightarrow \psi \rightarrow
%   \phi$ & \\
%   (\refstepcounter{rownum}\arabic{rownum}) & $\vdash \phi \rightarrow \top$ & \\
 & & \\
  (\refstepcounter{rownum}\arabic{rownum}) & $\vdash \Box(\phi \rightarrow \psi) \rightarrow \Box \phi \rightarrow
  \Box \psi$ & Axiom $K$\\
& & \\
  (\refstepcounter{rownum}\arabic{rownum}) & $\vdash \Box \phi \rightarrow \phi$ & \multirow{3}{8.5cm}{$R$
    is an equivalence relation}\\
  (\refstepcounter{rownum}\arabic{rownum}) & $\vdash \Box \phi \rightarrow \Box\Box \phi$ & \\
  (\refstepcounter{rownum}\arabic{rownum}) & $\vdash \diamondsuit \phi
  \rightarrow \Box \diamondsuit \phi$ & \\
& & \\ 
(\refstepcounter{rownum}\arabic{rownum}) &
$\vdash I \gamma \rightarrow \Box I \gamma$ & If $\gamma \in Y
  (w)$ and $w R v$ then $\gamma \in Y (v)$\\
& & \\
(\refstepcounter{rownum}\arabic{rownum}) &
$\vdash L \rho \rightarrow \Box L \rho$ & If $\rho \in Z (w)$
  and $w R v$ then $\rho \in Z (v)$ \\
& & \\
  (\refstepcounter{rownum}\arabic{rownum}) & $\vdash I \gamma
  \rightarrow \gamma$ & If $\gamma \in Y (w)$ then
  $\mathbbm{V}, w \Vdash_{VQL} \gamma$\\
& & \\
(\refstepcounter{rownum}\arabic{rownum}) &$\vdash L \rho
  \rightarrow \tmop{TR} (\rho)$ &  If $\rho \in Z (w)$ then $\mathbbm{V}, w \Vdash_{VQL} \tmop{TR} (\rho)$\\
% (\refstepcounter{rownum}\arabic{rownum}) & $\vdash \BBI_X (\phi \to \psi) \to \BBI_X \phi \to \BBI_X \psi$ & \\[6pt]
& & \\
(\refstepcounter{rownum2}\Roman{rownum2}) & 
 $\AxiomC{$\vdash \phi \to \psi$}
\AxiomC{$\vdash \phi$}
\BinaryInfC{$\vdash \psi$}
\DisplayProof$ & Modus Ponens\\[10pt]
(\refstepcounter{rownum2}\Roman{rownum2}) & 
 $\AxiomC{$\vdash \phi$}
\UnaryInfC{$\vdash \Box \phi$}
\DisplayProof$ & Necessitation\\
& & \\
% (\refstepcounter{rownum2}\Roman{rownum2}) \label{BMnec}& 
%  $\AxiomC{$\vdash \phi$}
% \UnaryInfC{$\vdash \BB_X \phi$}
% \DisplayProof$ &  \\
% (\refstepcounter{rownum2}\Roman{rownum2}) &
%  $\AxiomC{$\vdash \phi$}
% \UnaryInfC{$\vdash \BBI_X \phi$}
% \DisplayProof$ & \\[10pt]
\hline
\end{tabularx}
\caption{Axioms for $VQL$}
\label{table:VQLaxioms}
\end{table}

\begin{definition}[Model Deduction Update]
For a $VQL$ Kripke structure $\mathbb{V} = \langle W, R, V, Y,
Z\rangle$, and let $\rho \in \mathcal{R}(\Phi)$ be a rule.  Let
$\Gamma$ be the premises of $\rho$ and let $\gamma$ be the conclusion.
 Define $\mathbb{V}_\rho:=\langle W, R, V, Y',
Z\rangle$ where:
\[ Y'(w) := \begin{cases} Y(w) \cup \{\gamma\} & \textup{if $\Gamma
    \subseteq Y(w)$ and $\rho \in Z(w)$} \\
Y(w) & \textup{o/w}
\end{cases}\]
\end{definition}

\begin{lemma}
If $\mathbb{V}$ is a $VQL$ Kripke structure then so is
$\mathbb{V}_\rho$ where $\rho \in \mathcal{R}(\Phi)$
\end{lemma}
\begin{proof}
This is Proposition 1 in \cite{velzquez-quesada_inference_2009}.
\end{proof}

The idea of the above operation is that at every world it adds in the
conclusion of $\rho$ if the agent is aware of the premises of
$\rho$. Furthermore, it preserves the various properties that are imposed on $VQL$ structures.  We may
define syntax that corresponds with the above semantics in the
following fashion, by introducing a deduction update modality:

\begin{definition}
Define $Pre : \mathcal{R}(\Phi) \to \mathcal{L}_0(\Phi)$ as follows:
\[ Pre(\rho) := \bigwedge_{\psi \in \Gamma} I\psi \wedge L \rho \]
Where $\Gamma$ is the set of premises of the rule $\rho$.
\end{definition}

\begin{definition}
The grammar $\mathcal{VQLD}(\Phi)$ extends $\mathcal{VQL}(\Phi)$ and
is given as follows:
\[ \phi ::= p \in \Phi\ |\ \top\ |\ \phi \vee
\psi \ |\ \neg \phi \ |\ \Pos \phi  \ |\ I\gamma\ |\ L \rho\ | \
\langle D_\rho\rangle \phi \]
As before, $\gamma \in \mathcal{L}_0(\Phi)$, while $\rho \in
\mathcal{R}(\Phi)$.  Along with the usual abbreviations,  let $[D_\rho]
\phi$ abbreviates $\neg\langle D_\rho\rangle \phi$.
\end{definition}

We have the following semantics of the above syntax:

\begin{definition}
We may extend $\Vdash_{VQL}$ to the full $\mathcal{VQLD}(\Phi)$
grammar by stipulating:
\[ \mathbb{V},w \Vdash_{VQL} \langle D_\rho\rangle \phi \iff
\mathbb{V},w \Vdash_{VQL} Pre(\rho)\ \&\ \mathbb{V}_{\rho},w \Vdash_{VQL} \phi\]
\end{definition}

Every formula in $\mathcal{VQLD}(\Phi)$ is equivalent to a formula in
$\mathcal{VQL}(\Phi)$.  This can be seen by observing a series of
reduction axioms, exhibited in Table \ref{table:VQLredaxioms}.  

\begin{table}
\centering
%\newcounter{rownum}
\setcounter{rownum}{0}
%\newcounter{rownum2}
\setcounter{rownum2}{0}
\begin{tabular}{|lll|}
\hline
& & \\
  (\refstepcounter{rownum}D\arabic{rownum})&$ \vdash \langle D_\rho
  \rangle \top \IFF Pre(\rho)$ & \\
  (\refstepcounter{rownum}D\arabic{rownum})&$ \vdash \langle D_\rho
  \rangle p \IFF Pre(\rho)\wedge p$ &\\
  (\refstepcounter{rownum}D\arabic{rownum})&$ \vdash \langle D_\rho
  \rangle \neg \phi \IFF Pre(\rho)\wedge \neg\langle D_\rho \rangle
  \phi$& \\
  (\refstepcounter{rownum}D\arabic{rownum})&$ \vdash \langle D_\rho
  \rangle (\phi \vee \psi) \IFF \langle D_\rho
  \rangle \phi \vee \langle D_\rho
  \rangle \psi $& \\
  (\refstepcounter{rownum}D\arabic{rownum})&$ \vdash \langle D_\rho
  \rangle \Pos \phi \IFF Pre(\rho) \wedge \Pos \langle D_\rho \rangle
  \phi$& \\
  (\refstepcounter{rownum}D\arabic{rownum})&$ \vdash  \langle D_\rho
  \rangle I \gamma \IFF Pre(\rho)$& if $\gamma$ is the conclusion of
  $\rho$\\
  (\refstepcounter{rownum}D\arabic{rownum})&$ \vdash  \langle D_\rho
  \rangle I \gamma \IFF Pre(\rho) \wedge I \gamma$& if $\gamma$ is
  \textbf{not} the conclusion of $\rho$\\
  (\refstepcounter{rownum}D\arabic{rownum})&$ \vdash  \langle D_\rho
  \rangle L \sigma \IFF Pre(\rho) \wedge L \sigma$& \\
(D\refstepcounter{rownum2}\Roman{rownum2}) & 
 $\AxiomC{$\vdash \phi$}
\UnaryInfC{$\vdash [D_\rho] \phi$}
\DisplayProof$ & Necessitation\\
& & \\
\hline
\end{tabular}
\caption{Reduction Axioms for $\langle D_\rho \rangle \phi$}
\label{table:VQLredaxioms}
\end{table}

Below is a central result present in
\cite{velzquez-quesada_inference_2009}:
\begin{definition}
Let 
\[ \Gamma \vdash_{VQL} \phi \]
If and only if there is some $\Delta \subseteq_\omega \Gamma$ such
that $\bigwedge \Gamma \to \phi$ is derivable using the axioms and
rules present in
Tables \ref{table:VQLaxioms} and \ref{table:VQLredaxioms}
\end{definition}

\begin{theorem}[$VQL$ Strong Soundness and Completeness]
\[ \Gamma \vdash_{VQL} \phi \iff \Gamma \Vdash_{VQL} \phi \]
\end{theorem}
\begin{proof}
This is Theorem 3 of \cite{velzquez-quesada_inference_2009}.
\end{proof}

In \cite{velzquez-quesada_inference_2009} extends the above to 
discuss logical dynamics, where the agent can become aware of new rules.

To summarize, $VQL$ is a logic where inference is modeled using update
modalities, and where one is aware of a proposition only if it is a
fact they implicitly know.  The only rules that an agent is permitted to
be aware of are sound.  The agent can infer new things that they
already implicitly knew only after an update.

There is at least two points of commonality between $VQL$ and
\textsc{EviL}.  Perhaps the most notable is that both frameworks
assume that agents can only perform \emph{sound} reasoning.  Likewise,
both present a notion of implicit knowledge: in \textsc{EviL},
implicit or background knowledge can be interpreted as the 
universal modality $U$ presented in \S\ref{supersystems}, which embodies the background theory of the
model that the multiverse where we are modeling \textsc{EviL} agents
to dwell; the $\Box$ modality plays an analogous
r\^{o}le in $VQL$.

On the other hand, there are more ways in which \textsc{EviL} is different from $VQL$ then
there are ways the two are similar.  In $VQL$, agents can only be 
aware of things which they already implicitly
knew.  In \textsc{EviL}, agents can be fallible in their beliefs, and
believe everything they implicitly know 
(this follows from $\vdash U \phi \to \Box \phi$).
Likewise, the notion of knowledge as ``the existence of a sound
argument'' is very separate from \emph{implicit knowledge}.  In \textsc{EviL}, implicit knowledge $U\phi$ does not imply the existence of a sound
argument, for perhaps the \textsc{EviL} agent has not one single true
assumption to employ as a premise in any argument.  Neither is the
converse true, since holding a sound argument for $\phi$ in a
particular situation does not mean that it is true in every 
conceivable universe.

 In \textsc{EviL}, agents are assumed
to be logically omniscient, with all of the deductive powers of
\textsc{EviL} itself at their disposal.  In \textsc{EviL}, we the 
modelers must surrender some of our control
over how \textsc{EviL} agents are reasoning because of this.  In the
case of $VQL$, the agent is modeled as making inferences using 
very explicit syntax.  This allows $VQL$ far more fine grained control
over the inferences agents make than \textsc{EviL}, which is
\emph{laissez faire} in comparison.  It is herein that the key
difference between the approaches is apparent: in \textsc{EviL} one
can reason intuitively about what sort of conclusions agents must
arrive at, while in $VQL$ we can only conclude that the agent will
arrive at the sound conclusions we allow them to conclude.
In $VQL$, the agent is at the mercy of the modeler, and in
\textsc{EviL} it is more like the modeler is at the mercy of 
the \textsc{EviL} agents themselves.


Finally, $VQL$ and \textsc{EviL} draw from different philosophical
foundations. In $VQL$, one takes a conservative perspective on what it means for an
agent to know something in line with traditional epistemic logic.  As
we first saw in \S\ref{explicit}, the \textsc{EviL} perspective is more
closely related to foundationalism.

% Moreover, the update perspective present in $VQL$ is at odds with the
% inherently \textsc{EviL} structural perspective in \textsc{EviL}.  The
% idea behind $VQL$ and is to afford the logical accessibility relation
% as must freedom as possible, while the \textsc{EviL} approach is to
% fundamentally reduce accessibility relations to definitions within
% more concrete structures.  The next framework, $DAL$, makes this
% difference clear.

\subsection{Dynamic Awareness Logic}

\emph{Dynamic Awareness Logic} $(DAL)$ is the logic introduced in
\cite{van_benthem_inference_2009}.  As in the case of $VQL$, $DAL$ is
an original name we offer; the authors do not offer a name for their
work themselves.  It can be understood as modification on $VQL$ as we
previously
 discussed and  van Benthem's \emph{acts of realization}
\cite{van_benthem_inference_2009}.  Unlike $VQL$, which is intended to
model implicit and explicit knowledge, $DAL$ geared towards providing
a framework with fewer restrictions and easy to describe dynamics. 

We will focus on single agent $DAL$, however multi-agent $DAL$ is also
introduced in \cite{van_benthem_inference_2009}.  Likewise, $DAL$ also
accommodates \emph{action model} updates as introduced in
\cite{baltag_logic_1998}.  Since our purpose in investigating this
framework is to understand how it accommodates explicit inference, we
have chosen not to pursue these topics here.

We begin by reviewing the grammars employed in $DAL$:
\begin{definition}
Define the static awareness language $\mathcal{AL}(\Phi)$ as: 
\[ \phi ::= p \in \Phi\ |\ \neg \phi \ |\ \phi \wedge \psi \ |\ \Nec \phi
\ |\ I \phi \]

The dynamic awareness language $\mathcal{DAL}(\Phi)$ is defined as follows:
\[ \phi ::= p \in \Phi\ |\ \neg \phi \ |\ \phi \wedge \psi \ |\ \Nec \phi
\ |\ I \phi\ |\ [+\chi] \phi \ |\ [-\chi] \phi \]
\end{definition}

Note that the syntax introduced in \cite{van_benthem_inference_2009}
also contains \emph{public announcements}, as introduced in
\cite{gerbrandy_bisimulationsplanet_1998}.  Public announcements
correspond to factive updates a model can undergo.  Since this is the
textbook example of dynamics for modal logic, discussed at length in
\cite{ditmarsch_dynamic_2007} for instance, we have decided to eschew
discussion of it in favor of the more novel dynamics embodied by the
awareness update modalities $[+\chi]$ and $[-\chi]$. 

The structures employed in this
framework are similar to $VQL$ structures, but with less structure:
\begin{definition}
We call a structure $\mathbb{A} = \langle W, R, V, A\rangle$ is a
\textbf{Kripke Awareness Structure} over a language $\mathcal{L}$ and set of
proposition letters $\Phi$ when
\begin{bul}
  \item $W$ is a set of worlds
  \item $R \subseteq W \times W$ is an accessibility relation
  \item $V: \Phi \to \powerset W$ is a propositional valuation
    function
  \item $Y: \mathcal{L} \to \powerset W$ is a propositional awareness relation
\end{bul}
\end{definition}

Along with the above semantics, we also have two update operations:
\begin{definition}
Let $\mathbb{A} = \langle W, R, V, Y\rangle$ be a Kripke awareness
structure.

Define $\mathbb{A}_{+\chi} := \langle W, R, V, Y'\rangle$ where
\[ Y'(\phi) := \begin{cases} W & \textup{if $\phi = \chi$} \\
Y(\phi) & \textup{o/w} \end{cases} \]

Define $\mathbb{A}_{-\chi} := \langle W, R, V, Y'\rangle$ where
\[ Y'(\phi) := \begin{cases} \varnothing & \textup{if $\phi = \chi$} \\
Y(\phi) & \textup{o/w} \end{cases} \]
\end{definition}

These two operations correspond to adding and subtracting the awareness of
$\chi$ at all worlds.
This allows us to present semantics for the
language $\mathcal{DAL}(\Phi)$ as follows:

\begin{definition}
$DAL$'s truth predicate $\Vdash_{DAL}$ is defined such that:
\begin{eqnarray}
  \mathbb{A}, w \Vdash_{DAL} p & \Longleftrightarrow & w \in V (p)
  \nonumber\\
  \mathbb{A}, w \Vdash_{DAL} \neg \phi & \Longleftrightarrow &
  \mathbb{A}, w \nVdash_{DAL} \phi \nonumber\\
  \mathbb{A}, w \Vdash_{DAL} \phi \wedge \psi & \Longleftrightarrow &
  \mathbb{A}, w \Vdash_{DAL} \phi \hspace{1em} \& \hspace{1em}
  \mathbb{A}, v \Vdash_{DAL} \phi \nonumber\\
  \mathbb{A}, w \Vdash_{DAL} \Box \phi & \Longleftrightarrow & \forall
  v \in W.w R v \Longrightarrow \mathbb{A}, v \Vdash_{DAL} \phi
  \nonumber\\
  \mathbb{A}, w \Vdash_{DAL} I \phi & \Longleftrightarrow & \phi \in Y
  (w) \nonumber\\
  \mathbb{A}, w \Vdash_{DAL} [+ \chi] \phi & \Longleftrightarrow &
  \mathbb{A}_{+ \chi}, w \Vdash_{DAL} \phi \nonumber\\
  \mathbb{A}, w \Vdash_{DAL} [- \chi] \phi & \Longleftrightarrow &
  \mathbb{A}_{- \chi}, w \Vdash_{DAL} \phi \nonumber
\end{eqnarray}
\end{definition}

In \cite{van_benthem_inference_2009}, the authors do not provide a
calculus for $DAL$ and exhibit a completeness theorem.  Instead, they
employ the observation that every formula in $\mathcal{DAL}(\Phi)$ is logically
equivalent to a formula in $\mathcal{AL}(\Phi)$ for all $DAL$ Kripke
structures, and provide reduction axioms.  Hence when reasoning about this system one can always
reason about the static language without loss of generality to the
dynamic language.

\begin{theorem}
The valid formulas of the dynamic-epistemic awareness
language $\mathcal{DAL}(\Phi)$ are the valid
formulae of the static base language $\mathcal{AL}(\Phi)$ plus the reduction
axioms and modal inference rules listed in Table
\ref{table:dalredaxioms}.
\end{theorem}
\begin{proof}
This is Theorem 1 in \cite{van_benthem_inference_2009}.
\end{proof}

\begin{table}
\framebox[\linewidth]{
%\hspace{-0.5cm}
%\vspace{-0.6cm}
\begin{minipage}[h]{0.45\linewidth}
$ \begin{array}{llll}
     \vdash [+ \chi] p & \leftrightarrow & p \\
     \vdash [+ \chi] I \chi & \leftrightarrow & \top \\
     \vdash [+ \chi] I \phi & \leftrightarrow & I \phi \ \ \ \tmop{for} \phi \neq
     \chi \\
     \vdash [+ \chi] \neg \phi & \leftrightarrow & \neg [+ \chi] \phi \\
     \vdash [+ \chi] (\phi \wedge \psi) & \leftrightarrow & [+ \chi] \phi
     \wedge [+ \chi] \psi \\
     \vdash [+ \chi]\Box \phi & \leftrightarrow & \Box[+ \chi] \phi \\
     \vdash \phi & \Longrightarrow & \vdash [+\chi] \phi
%      \\
% & & &
   \end{array} $
\end{minipage}
\hspace{0.5cm}
\begin{minipage}[h]{0.45\linewidth}
$ \begin{array}{llll}
     \vdash [- \chi] p & \leftrightarrow & p \\
     \vdash [- \chi] I \chi & \leftrightarrow & \bot \\
     \vdash [- \chi] I \phi & \leftrightarrow & I \phi \ \ \ \tmop{for} \phi \neq
     \chi \\
     \vdash [- \chi] \neg \phi & \leftrightarrow & \neg [- \chi] \phi \\
     \vdash [- \chi] (\phi \wedge \psi) & \leftrightarrow & [- \chi] \phi
     \wedge [- \chi] \psi \\
     \vdash [- \chi]\Box \phi & \leftrightarrow & \Box[- \chi] \phi \\
     \vdash \phi & \Longrightarrow & \vdash [- \chi] \phi
%      \\
% & & &
   \end{array} $
\end{minipage}\\ 
\\
}
\caption{Reduction Axioms for $DAL$}
\label{table:dalredaxioms}
\end{table}

The authors in \cite{van_benthem_inference_2009}
an intuitive motivation of how their framework is intended to model
explicit knowledge.  Informally, we will in this setting we will want
to employ the following readings 
\begin{bul}
\item  $\Box\phi$ asserts ``the agent \emph{implicitly} knows that
$\phi$.''
\item $K\phi$ asserts ``the agent \emph{explicitly} knows that
$\phi$.''
\end{bul}
For an agent that is not logically omniscient, we
want to enforce that the following is \textbf{not} valid:
\begin{equation}
 K(\phi \to \psi) \to K \phi \to K \psi \label{eq:not-valid}\end{equation}

However, while the above does not hold for $DAL$, the authors in
\cite{van_benthem_inference_2009} suggest that the following two
expressions should hold:
\begin{eqnarray*}  
& \vdash K(\phi \to \psi) \to K \phi \to \Box \psi \\
&  \vdash K(\phi \to \psi) \to K \phi \to [?] K \psi 
\end{eqnarray*}
Here $[]$ is an inference \emph{gap}.  The idea here is that agents
are thought to implicitly know the logical conclusions of all of their
explicit knowledge, while to explicitly draw a conclusion they need a nudge.

The authors propose the following definitions, and illustrate that
they are sufficient to ensure the above criteria:
\begin{eqnarray*}  
K \phi & := & \Box (\phi \wedge I \phi)  \\
\ [?] K \psi & := & [+\psi] K \psi
\end{eqnarray*}
As a side remark, the gap $[?]$ is defined to be sensitive to the context in
which is it employed.

The proposed definition for explicit knowledge also renders
\eqref{eq:not-valid} for $VQL$.  In addition, the proposed
\textsc{EviL} formulation of knowledge $K_{\textup{\textsc{EviL}}} \phi :=
\DM(\PP \wedge \Nec \phi)$ also invalidates \eqref{eq:not-valid},
despite the fact that agents are modeled as logically omniscient.
This is due to the fact that the \textsc{EviL} agent may have a sound
argument for $\phi \to \psi$ and another, entirely different sound
argument for $\phi$, but she might not be clever enough to join the
bases for these two sound arguments together to argue that $\psi$.  It
is straightforward to compose a 
concrete \textsc{EviL} model that formalizes this intuition.  Failure
of explicit logical omniscience is a unifying principle
behind all formulations of explicit knowledge.
% Formally, this sort of behavior is exhibited in the following concrete
% \textsc{EviL} model:
% \[ \mathfrak{M}:= \{(\varnothing,\varnothing), (\{p\},\varnothing), (\{p,q\},\{p }), (\{p,q\},\{p \to q}),\]

While $VQL$ naturally accommodates $DAL$'s formulation of explicit
knowledge, it cannot accommodate $DAL$'s gap $[?]$.   $DAL$ has a
richer internal language that is unavailable to $VQL$.  While $VQL$ supports dynamics for adding awareness of
formulae, the agent can only be aware of $\mathcal{L}_0(\Phi)$
formulae.  This grammar restriction prohibits formulating the dynamics
above.

The gap $[?]$ is also incompatible with \textsc{EviL}.  As previously noted,
\textsc{EviL} is restricted to $\mathcal{L}_0$ in a manner similar to
$VQL$.  
Another 
issues is that the abstract \textsc{EviL} semantics makes no mention of 
anything analogous to explicit awareness, and the concrete semantics,
while superficially similar, is sensitive to dynamics in a way that
awareness approaches are not.  For instance, in the concrete
\textsc{EviL} semantics
we evidently have that:
$$(a,\{p\}) \not\sqsubseteq (a,\{q\}).$$  On the other hand, we have that
$$(a,\{p\}\cup \{p\}) \sqsubseteq (a,\{q\}\cup \{p\}).$$
This poses a problem for trying to formulate an \textsc{EviL}
formulation of $[+\chi]$.  Updates conventionally do not \emph{add}
accessibility, while in the concrete semantics they would have to in
this situation.  Given this observation, it is straightforward to
contrive a scenario in the concrete \textsc{EviL} semantics which
invalidates the following candidate reduction axiom:
\[ [+\chi] \BP \phi \IFF \BP [+\chi] \phi \]

The above reveals a tension between the dynamic approach and the
\textsc{EviL} framework.  In \textsc{EviL} concrete models, two worlds
are related if their beliefs and truth conditions make true various
logical relationships, while in a dynamic approach two worlds are related if
they were initially related, and through various manipulations to the model
this relationship has been preserved.

As a final remark concerning $DAL$, when one writes $\mathbb{A},w
\vdash K \phi$, it is not obvious what sort of reasons the agent might
have for knowing $\phi$ in this circumstance.  The story in $VQL$ is
that the agent used valid inference rules they were aware of, along
with various model updates to arrive at $\phi$.  As we have already
presented, \textsc{EviL} has its own way of analyzing the reasoning
behind knowledge agents might or might not have. However, no such
narrative appears to be evident in the case of $DAL$.  However, given
the expressive power of $DAL$, perhaps $DAL$ will be able to
embed other logics of explicit inference.  The character of $DAL$ is
that it is general, and non-specific regarding the way that it models
explicit knowledge, so understanding its precise connection to other
approaches seems a promising subject for future research.

\subsection{Final Thoughts}

The purpose of all of the logics we have presented in this section is
to model an agent making explicit inferences behind their deductions.
Awareness based approaches tackle this by modeling the mental
contents of the agent's mind.  At a glance, this appears superficially
similar to the \textsc{EviL} approach.  However, at their core, these
two approaches are essentially orthogonal.

To summarize the \textsc{EviL} approach, it is to enforce that beliefs
correspond to deductions, and then provide modalities for controlling 
the bases upon which those deductions are made.  It accomplishes this
by introducing concrete structures with modal syntax, and then
abstracting on those concrete structures with abstract Kripke
semantics.

The cost of this approach, however, is that the intended semantics
for controlling the bases for inference are not obviously compatible
with the intended semantics for updates.

However, there are benefits to the \textsc{EviL} approach that
outweigh the costs.  \textsc{EviL} incorporates a clear
foundationalist epistemology into its design, and naturally models 
deliberative actions one may take in a foundationalist setting. 
This foundationalist perspective on belief and knowledge clear intuitionistic readings, and
this reflected in embedding.  However, via \emph{duality},
\textsc{EviL} goes further and introduces its own reading of
intuitionistic semantics, where instead imagination takes the place of
knowledge as providing truth conditions.

We hope that the future will allow for fruitful communication
between the ideas in \textsc{EviL} and its allied approaches.

% \subsection{Justification Logic}

% \emph{Justification Logic} is perhaps the most advanced out of all of
% the frameworks presented.  The logic evolved out of earlier work on
% the \emph{Logic of Proofs} \cite{artemov_logic_1994}.  Historically
% the \emph{Logic of Proofs} has been understood as follows:
% \begin{quote}
% The \emph{Logic of Proofs} ($LP$) bears the same relationship to explicit proofs in formal arithmetic that
% \emph{G\"{o}del-L\"{o}b logic} (GL) bears to arithmetic provability.\cite{fitting_logic_2005}
% \end{quote}

% \begin{definition}
% The syntax of $LP$ conforms to the following grammar, which we will
% call $\mathcal{LP}(\Phi,\mathcal{C})$:
% \[ \phi ::= p \in \Phi \ |\ \bot \ | \ \phi \to \psi\ | \ t: \phi  \]
% Here $t$ is referred to as a \textbf{proof polynomial}, and has a
% separate grammar which gives its own composition, over a set of
% \textbf{constants} $\mathcal{C}$.  The grammar of proof polynomials
% $\mathcal{PP}(\mathcal{C})$ is given as:
% \[ t ::= c \in \mathcal{C} \ |\ t \cdot u \ |\ t + u \ |\ ! t \]
% \end{definition}

% The intuition behind $t:\phi$ is that $t$ is a structure that contains
% the proofs of several propositions, and $\phi$ is one of the
% propositions it proves.




% We can see that the above Kripke structures can be understood from a
% more \textsc{EviL} perspective, where we eschew the awareness function
% in favor of having world/belief set pairs:

% \begin{definition}
% We call a structure $\mathfrak{A} = \langle U, R, V\rangle$ is a \textbf{\textsc{EviL}
%   Awareness Kripke Structure} over a language $\mathcal{L}$, a set of
% proposition letters $\Phi$ and a set of worlds $W$ when
% \begin{bul}
%   \item $U\subseteq W \times \powerset\mathcal{L}$ is a set of pairs
%     of worlds and sets of formulae in the language $\mathcal{L}$
%   \item $R \subseteq W \times W$ is an accessibility relation
%   \item $V: \Phi \to \powerset W$ is a propositional valuation
%     function
% \end{bul}
% \end{definition}

% \subsection{Comparing Frameworks}
% In this section, we introduce a series of philosophical and technical issues that
% the disparate approaches address, and how their approaches differ.
% Briefly, these issues are:

% \begin{description}
%   \item[Multiple Agents] -- Does the framework support multiple
%     agents?
%   \item[Basic Epistemic] -- How do the frameworks model: \emph{implicit
%       belief}, \emph{explicit belief}, \emph{implicit knowledge},
%     and \emph{explicit knowledge}?
%   \item[Implicit/Explicit Knowledge Relationship] -- What is the
%     relationship that implicit knowledge bares to explicit knowledge?
%   \item[Dynamics] -- What kind of dynamics can the framework model?
% %   \item[Inference Gap] --  For logics that model agents that are not
% %     logically omniscient, there is an inference gap:
% % \[ Ex (\phi \to \psi) \to Ex \phi \to [] Ex \psi \]
% % How does the framework fill the gap?
%   \item[Grammar Restrictions] -- If there are grammar restrictions,
%     why are they imposed?
%   \item[Semantics] -- What kind of semantics does the framework
%     support?
%   \item[Explicit Inference] -- How does the framework model explicit inference?
%   \item[Intuitionistic Logic] -- Does the framework present a
%     connection to intuitionistic logic?
%   \item[Complexity] -- What are complexity results for the framework?
% \end{description}

% \begin{table}
% \centering
% %\newcounter{rownum}
% \setcounter{rownum}{0}
% %\newcounter{rownum2}
% \setcounter{rownum2}{0}
% \begin{tabular}{|ll|}
% \hline
%   (\refstepcounter{rownum}\arabic{rownum})&$ \vdash \phi \to \psi \to \phi$\\
%    (\refstepcounter{rownum}\arabic{rownum})&$ \vdash (\phi \to \psi
%    \to \chi) \to (\phi \to \psi) \to \phi \to \chi$\\
%    (\refstepcounter{rownum}\arabic{rownum})&$ \vdash \phi \wedge \psi \to
%    \phi$\\
%  (\refstepcounter{rownum}\arabic{rownum})&$ \vdash \phi \wedge \psi \to
%  \psi$\\
%  (\refstepcounter{rownum}\arabic{rownum})&$ \vdash (\phi \to \psi)
%  \to (\phi \to \chi) \to (\phi \to \psi \wedge \chi)$\\
%  (\refstepcounter{rownum}\arabic{rownum})&$ \vdash \phi \to \phi \vee \psi$\\
%  (\refstepcounter{rownum}\arabic{rownum})&$ \vdash \psi \to \phi \vee
%  \psi$\\
%  (\refstepcounter{rownum}\arabic{rownum})&$ \vdash (\phi \to \chi)
%  \to (\psi \to \chi) \to (\phi \vee \psi) \to \chi$\\
%  (\refstepcounter{rownum}\arabic{rownum})&$ \vdash \bot \to \phi$\\
% (\refstepcounter{rownum2}\Roman{rownum2}) & \AxiomC{$\vdash \phi \to
%   \psi$} \AxiomC{$\vdash \phi$} \BinaryInfC{$\vdash \psi$}
% \DisplayProof \\ % & Modus Ponens\\%[10pt]
% \hline
% \end{tabular}
% \caption{$VQL$ Systems}
% \label{table:VQaxioms}
% \end{table}

%%% Local Variables: 
%%% mode: latex
%%% TeX-master: "evil_philosophy"
%%% End: 
