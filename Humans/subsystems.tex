In this section, investigate subsystems and super-systems of
\textsc{EviL}.

In addition to the main system presented above, it can be understood to contain two subsystems, corresponding to two fragments of the main grammar:
\begin{definition}
Define $\mathcal{L}^\boxminus (\Phi, \mathcal{A})$ as the fragment:
\[ \phi \ {::=} \  p \in \Phi \  | \  \phi
   \rightarrow \psi \  | \  \bot \  |
   \  \Box_X \phi \  | \  \boxminus_X \phi
%   \  | \  \boxplus_X \phi
 \  | \ 
   \circlearrowleft_X \]

And define $\mathcal{L}^\boxplus (\Phi, \mathcal{A})$ as the fragment:
%Define $\mathcal{L}^\boxminus (\Phi, \mathcal{A})$ as the fragment:
\[ \phi \ {::=} \  p \in \Phi \  | \  \phi
   \rightarrow \psi \  | \  \bot \  |
   \  \Box_X \phi 
%\  | \  \boxminus_X \phi
   \  | \  \boxplus_X \phi
 \  | \ 
   \circlearrowleft_X \]
\end{definition}

Table \ref{table:axiomsII} gives the axioms systems for these two fragments.  For now, we shall observe that \textsc{EviL} extends \textsc{EviL}$^\BM$ and \textsc{EviL}$^\BP$.  In \S\ref{conservative-extension} we shall make this precise. 
\begin{table}
\begin{minipage}[b]{0.5\linewidth}
\centering
%\newcounter{rownum}
\setcounter{rownum}{0}
%\newcounter{rownum2}
\setcounter{rownum2}{0}
\begin{tabular}{|ll|}
\hline
  (\addtocounter{rownum}{1}\arabic{rownum})&$ \vdash \phi \rightarrow \psi \rightarrow \phi$\\
  (\addtocounter{rownum}{1}\arabic{rownum})&$ \vdash (\phi \rightarrow \psi \rightarrow \chi) \rightarrow (\phi
  \rightarrow \psi) \rightarrow \phi \rightarrow \chi$\\
  (\addtocounter{rownum}{1}\arabic{rownum})&$ \vdash (\neg \phi \rightarrow \neg \psi) \rightarrow \psi \rightarrow
  \phi$\\
  (\addtocounter{rownum}{1}\arabic{rownum})&$ \vdash \boxminus_X \phi \rightarrow \phi$\\
  (\addtocounter{rownum}{1}\arabic{rownum})&$ \vdash \boxminus_X \phi \rightarrow \boxminus_X \boxminus_X \phi$\\
  (\addtocounter{rownum}{1}\arabic{rownum})&$ \vdash p \rightarrow \boxminus_X p$\\
  (\addtocounter{rownum}{1}\arabic{rownum})&$ \vdash \neg p \rightarrow \boxminus_X \neg p$\\
  (\addtocounter{rownum}{1}\arabic{rownum})&$ \vdash \diamondsuit_X \phi \rightarrow \boxminus_X \diamondsuit_X \phi$\\
  (\addtocounter{rownum}{1}\arabic{rownum})&$ \vdash \Box_X \phi \rightarrow \Box_X \boxminus_Y \phi$\\
  (\addtocounter{rownum}{1}\arabic{rownum})&$ \vdash \phi \rightarrow \boxminus_X (\circlearrowleft_X \rightarrow
  \diamondsuit_X \phi)$\\
  (\addtocounter{rownum}{1}\arabic{rownum})&$ \vdash \circlearrowleft_X \rightarrow \boxminus_X \circlearrowleft_X$\\
  (\addtocounter{rownum}{1}\arabic{rownum})&$ \vdash \Box_X (\phi \rightarrow \psi) \rightarrow \Box_X \phi \rightarrow
  \Box_X \psi$\\
  (\addtocounter{rownum}{1}\arabic{rownum})&$ \vdash \boxminus_X (\phi \rightarrow \psi) \rightarrow \boxminus_X \phi
  \rightarrow \boxminus_X \psi$\\
(\addtocounter{rownum2}{1}\Roman{rownum2}) & 
 $\AxiomC{$\vdash \phi \to \psi$}
\AxiomC{$\vdash \phi$}
\BinaryInfC{$\vdash \psi$}
\DisplayProof$ \\ %& Modus Ponens\\[10pt]
(\addtocounter{rownum2}{1}\Roman{rownum2}) & 
 $\AxiomC{$\vdash \phi$}
\UnaryInfC{$\vdash \Box_X \phi$}
\DisplayProof$ \\ %& \multirow{3}{8.5cm}{Variations on necessitation}\\
(\addtocounter{rownum2}{1}\Roman{rownum2}) & 
 $\AxiomC{$\vdash \phi$}
\UnaryInfC{$\vdash \BB_X \phi$}
\DisplayProof$   \\
% (\addtocounter{rownum2}{1}\Roman{rownum2}) &
%  $\AxiomC{$\vdash \phi$}
% \UnaryInfC{$\vdash \BBI_X \phi$}
% \DisplayProof$  \\% [10pt]
\hline
\end{tabular}
\end{minipage}
\hspace{0.5cm}
\begin{minipage}[b]{0.5\linewidth}
 \centering
%\newcounter{rownum}
\setcounter{rownum}{0}
%\newcounter{rownum2}
\setcounter{rownum2}{0}
\begin{tabular}{|ll|}
\hline
  (\addtocounter{rownum}{1}\arabic{rownum})&$ \vdash \phi \rightarrow \psi \rightarrow \phi$\\
  (\addtocounter{rownum}{1}\arabic{rownum})&$ \vdash (\phi \rightarrow \psi \rightarrow \chi) \rightarrow (\phi
  \rightarrow \psi) \rightarrow \phi \rightarrow \chi$\\
  (\addtocounter{rownum}{1}\arabic{rownum})&$ \vdash (\neg \phi \rightarrow \neg \psi) \rightarrow \psi \rightarrow
  \phi$\\
  (\addtocounter{rownum}{1}\arabic{rownum})&$ \vdash \boxplus_X \phi \rightarrow \phi$\\
  (\addtocounter{rownum}{1}\arabic{rownum})&$ \vdash \boxplus_X \phi \rightarrow \boxplus_X \boxplus_X \phi$\\
  (\addtocounter{rownum}{1}\arabic{rownum})&$ \vdash p \rightarrow \boxplus_X p$\\
  (\addtocounter{rownum}{1}\arabic{rownum})&$ \vdash \neg p \rightarrow \boxplus_X \neg p$\\
  (\addtocounter{rownum}{1}\arabic{rownum})&$ \vdash \Box_X \phi \rightarrow \boxplus_X \Box_X \phi$\\
  (\addtocounter{rownum}{1}\arabic{rownum})&$ \vdash \Box_X \phi \rightarrow \Box_X \boxplus_Y \phi$\\
  (\addtocounter{rownum}{1}\arabic{rownum})&$ \vdash \phi \rightarrow \boxplus_X (\circlearrowleft_X \rightarrow
  \diamondsuit_X \phi)$\\
  (\addtocounter{rownum}{1}\arabic{rownum})&$ \vdash \neg \circlearrowleft_X \rightarrow \boxplus_X \neg
  \circlearrowleft_X$\\
  (\addtocounter{rownum}{1}\arabic{rownum})&$ \vdash \Box_X (\phi \rightarrow \psi) \rightarrow \Box_X \phi \rightarrow
  \Box_X \psi$\\
  (\addtocounter{rownum}{1}\arabic{rownum})&$ \vdash \boxplus_X (\phi \rightarrow \psi) \rightarrow \boxplus_X \phi
  \rightarrow \boxplus_X \psi$\\
(\addtocounter{rownum2}{1}\Roman{rownum2}) & 
 $\AxiomC{$\vdash \phi \to \psi$}
\AxiomC{$\vdash \phi$}
\BinaryInfC{$\vdash \psi$}
\DisplayProof$ \\ %& Modus Ponens\\[10pt]
(\addtocounter{rownum2}{1}\Roman{rownum2}) & 
 $\AxiomC{$\vdash \phi$}
\UnaryInfC{$\vdash \Box_X \phi$}
\DisplayProof$ \\ %& \multirow{3}{8.5cm}{Variations on necessitation}\\
% (\addtocounter{rownum2}{1}\Roman{rownum2}) & 
%  $\AxiomC{$\vdash \phi$}
% \UnaryInfC{$\vdash \BB_X \phi$}
% \DisplayProof$   \\
(\addtocounter{rownum2}{1}\Roman{rownum2}) &
 $\AxiomC{$\vdash \phi$}
\UnaryInfC{$\vdash \BBI_X \phi$}
\DisplayProof$  \\% [10pt]
\hline
\end{tabular}
\end{minipage}
\caption{Axiom system \textsc{EviL}$^\BM$ and \textsc{EviL}$^\BP$ respectively}
\label{table:axiomsII}
\end{table}


%%% Local Variables: 
%%% mode: latex
%%% TeX-master: "evil_philosophy"
%%% End: 
