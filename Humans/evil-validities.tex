The previous philosophical readings of \textsc{EviL} immediately
suggest certain validities will hold the semantics.  For instance, the
assertion ``A set of premises is sound if and only if all of its
subsets are sound.'' would be expressed as
\begin{equation}
\VDash \PP \IFF \BM \PP \label{ppequiv}
\end{equation}
\ldots and indeed, this is a validity of \textsc{EviL}.  There is
another, related validity associated with $\PP$; namely that if the
\textsc{EviL} agent's assumptions are sound, then anything she
concludes from them is true (employing the reading which naturally
arises from Theorem \ref{theorem-theorem}).  This is expressed as
\begin{equation}
\VDash \PP \to \Box \phi \to \phi \label{axiom-11}
\end{equation}
The formula \eqref{ppequiv} expresses that the soundness of one's
premises  is something \emph{persistent} as the \textsc{EviL} agent
carries on casting doubt on assumptions and discarding them.  Another
thing that is persistent this way is the \text{EviL} agent's
imagination:
\begin{equation}
\VDash \Pos \phi \to \BM \Pos \phi \label{axiom-8}
\end{equation}
I read \eqref{axiom-8} as saying something like ``If the \textsc{EviL}
agent can imagine something, then no matter things she casts into
doubt, she can still imagine it.''  One can also express something
like the dual of this, namely
\begin{equation}
\VDash \Box \phi \to \BP \Box \phi
\end{equation}
\ldots which I read as asserting ``If th agent can compose an argument
then she'll still be able to compose that argument if she remembers
more premises she has available.''  In general, many of the assertions
here have an interplay like this -- interest in these relationships is
taken up in \S\ref{elimination}.

Furthermore, for better or for worse the \textsc{EviL} semantics make
true the following: if something is achievable by repeatedly casting
assumptions into doubt, then it's achievable by casting assumptions
into doubt only once:
\begin{equation}
\Vdash \PM^+\phi \to \PM \phi
\end{equation}
\ldots where $^+$ is taken from the syntax for \emph{regular
  expressions} commonly used in computer science meaning ``one or
more'' \citep{friedl_mastering_2006}.  Similarly, I have assumed that
discarding no assumptions is, in a way, vacuously casting assumptions
into doubt.  In light of this \textsc{EviL} also makes true the following:
\begin{equation}
\Vdash \phi \to \PM \phi
\end{equation}
%%% Local Variables: 
%%% mode: latex
%%% TeX-master: "evil_philosophy"
%%% End: 
