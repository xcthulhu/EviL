In this section, we show how \textsc{EviL} may be extended with a
universal modality $U$, as presented in \cite[chapter 7, pg
79]{van_benthem_modal_2010}.  Just as the previous section illustrated that
looking at fragments of \textsc{EviL} added complexity to the
completeness theorem, so too do natural extensions to the calculus.
We shall mention the relationship of the universal modality to
traditional epistemic logic, and discuss an analogue of the Theorem
Theorem that the universal modality obeys.

In this section, we shall provide sketches rather than the more extensive
proofs as we have so far provided.  This is because we really intend
for the results in this section to be minor modifications of our
previous results.  Our intention is to indicate what modifications are
to be made to accommodate our proposed extension.

The following gives the extended grammar of \textsc{EviL} with an added
universal modality:
$\mathcal{L}^U (\Phi,\mathcal{A})$ is the fragment:
\[ \phi \ {::=} \  p \in \Phi \  | \  \phi
   \rightarrow \psi \  | \  \bot \  |
   \  \Box_X \phi \  | \  \boxminus_X \phi
   \  \BP_X \phi \  | \  U \phi
 \  | \ 
   \circlearrowleft \]
It is important to note that our other fragments may be similarly extended.

Universal modality has the following semantics for Kripke structures: 
\[ \mathbb{M}, w \Vdash U \phi \iff \mathbb{M}, v \Vdash \phi\textup{
  for all $v \in W$} \]
Likewise, it has corresponding semantics for \textsc{EviL}
models:
\[ \Omega, w \VDash U \phi \iff \Omega, v \VDash \phi\textup{
  for all $v \in \Omega$} \]

Universal modality has its own associated analogue of the Theorem Theorem, which while
rather trivial, nonetheless allows us to understand the connection of
semantics of $U \phi$ to the logic of \textsc{EviL}:
\begin{proposition}
\[\Omega,(a,A) \VDash U \phi \iff Th(\Omega) \vdash \phi\]
\end{proposition}
Note that since $Th(\Omega)$ is necessarily closed under deduction,
then the above means that $\phi \in Th(\Omega)$.

Compare this with the original Theorem Theorem:
\[ \Omega,(a,A) \VDash \Box_X \phi \iff 
Th(\Omega) \cup A_X \vdash \phi \]

Previously, the
background knowledge $Th(\Omega)$ was implicit in the 
Theorem Theorem
reading of the semantics of $\Box_X \phi$; we can see now that $U$
makes these semantics explicit.

As mentioned in \cite[Chapter 7.4]{van_benthem_modal_2010},
universal modality is closely related to the modal logic $S5$. Below,
we state the axioms for the univesal modality in \textsc{EviL}, which
are recognizably the axioms for $S5$, along with other axioms asserting
that the other relations are subrelations of $U$.

\begin{table}
\centering
%\newcounter{rownum}
\setcounter{rownum}{0}
%\newcounter{rownum2}
\setcounter{rownum2}{0}
\begin{tabular}{|ll|}
\hline
  (\refstepcounter{rownum}U\arabic{rownum})&$ \vdash U
  (\phi \to \psi) \to U \phi \to U \psi$\\
   (\refstepcounter{rownum}U\arabic{rownum})&\label{Urefl}$ \vdash U \phi \rightarrow  \phi$\\
  (\refstepcounter{rownum}U\arabic{rownum})&$ \vdash U \phi \to U U \phi$\\
  (\refstepcounter{rownum}U\arabic{rownum})&\label{Uintro}$ \vdash \neg U \phi \to U
  \neg U \phi$\\
  (\refstepcounter{rownum}U\arabic{rownum})\label{BoxU}&$ \vdash U
  \phi \to \Box_X \phi$\\
  (\refstepcounter{rownum}U\arabic{rownum})\label{BMU}&$ \vdash U
  \phi \to \BM_X \phi$\\
  (\refstepcounter{rownum}U\arabic{rownum})\label{BPU}&$ \vdash U
  \phi \to \BP_X \phi$\\
(\addtocounter{rownum2}{1}\Roman{rownum2}) &
 $\AxiomC{$\vdash \phi$}
\UnaryInfC{$\vdash U \phi$}
\DisplayProof$  \\% [10pt]
\hline
\end{tabular}
\caption{Additional axiom and rules for the universal modality}
\label{table:Uaxioms}
\end{table}

We can think of the universal modality axioms (appropriately
restricted) as extending any of the
three systems we have looked at so far; \textsc{EviL} extends to
U\textsc{EviL},
\textsc{EviL}$^\BM$ extends to U\textsc{EviL}$^\BM$, and
\textsc{EviL}$^\BP$ extends to U\textsc{EviL}$^\BP$.  Abstract completeness
for all three systems is achieved in a similar manner.

\begin{theorem}[Universal \textsc{EviL} Soundness and Completeness]\label{universal-completeness}\ \\
\begin{align*}
\Gamma \vdash_{U\textup{\textsc{EviL}}} \phi & \iff \Gamma\Vdash_{\textup{\textsc{EviL}}}
\phi \\
\Gamma \vdash_{U\textup{\textsc{EviL}$^\BM$}} \phi & \iff \Gamma\Vdash_{\textup{\textsc{EviL}}}
\phi \\
\Gamma \vdash_{U\textup{\textsc{EviL}}^\BP} \phi & \iff \Gamma\Vdash_{\textup{\textsc{EviL}}}
\phi 
\end{align*}
\end{theorem}
\begin{proof}
Soundness in all cases is straightforward.

For completeness, in each case we carry out the canonical model
construction.
Note that axioms \ref{Urefl}--\ref{Uintro} enforce 
that the accessibility relation associated with $U$ forms
a partition on the canonical model, and at every point within a
given partition.  Axioms \ref{BoxU}--\ref{BPU} ensure the other relations are a subrelation of the
candidate universal relation.  In each case the canonical model construction will
provide a world $w$ which witnesses $\Gamma$ but does not witness
$\phi$.  To complete the construction, one need only take a \emph{point
  generated submodel} around $w$; see \cite[chapter
2, pg. 210]{blackburn_modal_2001} for a discussion on how ``A point
generated submodel suffices.''
This construction preserves the truth of all formulae at $w$ but 
establishes $U$ as a universal modality.

From there, it is straightforward to verify that all of the bisimilar
model completions we have investigated preserve the truth of $U \phi$
and $\neg U \phi$ at every world.
In each case, these bisimilar completions may be used just as before to establish the
abstract completeness theorem desired.
\end{proof}

Just as the above proof illustrates our previous abstract completeness
theorems may be adapted, our finitary approaches may be modified to
accommodate universal modality as well.

\begin{theorem}[Small Model Property for Universal \textsc{EviL}]\ \\
For any universal \textsc{EviL} formula $\phi$, if it satisfiable then
it is satisfiable in an \textsc{EviL} model with $O(EXP2(|\phi|))$
many worlds.
\end{theorem}
\begin{proof}
By assumptions and soundness, we know that $\nvdash \neg
\phi$, so we can make a finite model by using the finite canonical model construction
$\Cross$ we
previously saw in \S\ref{small-model}.  As with the abstract canonical
model construction, we shall ultimately want to take a point-generated
submodel around the world we constructed which witnesses $\phi$.

Just as in the original definition of $\Cross$, our finite model
construction needs a special definition for the universal modality.
Define the relation associated with the universal
modality as follows:
\[ w U v \iff (U \phi \in w \iff U \phi \in v)\]

Where $\mathbb{M}, w \Vdash U \phi \iff $ for all $v \in W$. 

Along with the axioms \ref{Urefl}--\ref{Uintro}, these will enforce
that $U$ forms an $S5$ modality; see 
\cite[chapter 5, pgs. 81--82]{boolos_logic_1995} for details.  

To ensure that the other relations are subrelations of $U$, 
one needs to ensure that $\Sigma$ is extended so the definition
includes the following:
\begin{align*}
\Sigma(\Delta,U \phi) := &   \{ U \phi, \neg U \phi \} \cup \\
& \{ \Nec_X \phi,\neg \Nec_X \phi, \\
& \ \BB_X \phi, \neg\BB_X \phi, \\
& \ \BBI_X \phi, \neg\BBI_X \phi  \ |\ X \in \Delta
\}
\end{align*}
Given $\Sigma$ constructed in this fashion, one may readily verify
that the Universal \textsc{EviL} axioms enforce the that other
relations are subrelations of our candidate Universal relation.  One
may then use the $\invis$ bisimulation to complete $\Cross$ from
partly \textsc{EviL} Kripke structure to a fully \textsc{EviL} Kripke structure.

Furthermore, we may note that the complexity of $\Sigma$ is unchanged by
this modification, so our previous bound of $O(EXP2(|\phi|))$ we gave
on the number of worlds in $\Cross$ and $\invis^\Cross$ do not change from what we
provided in Theorem \ref{small-model-property}.
\end{proof}

The above small model property has two consequences:

\begin{theorem}[$U\textup{\textsc{EviL}}$ Decidability]\label{uevil-decidability}\ 
$U\textup{\textsc{EviL}}$, $U\textup{\textsc{EviL}$^\BM$}$, and
$U\textup{\textsc{EviL}$^\BP$}$  are decidable and 
the time complexity of their decision problems is bounded above by $O(EXP3(|\phi|))$
\end{theorem}
\begin{proof}
The proof proceeds the same as the proof of Theorem
\ref{evil-decidability}, the \textsc{EviL} decidability theorem 
from \S\ref{small-model}.
\end{proof}

\begin{theorem}\label{UEviL-concrete}
Assuming that $\Gamma$ is finite and the set of letters $\Phi$ is universal:
\begin{align*}
\Gamma \vdash_{U\textup{\textsc{EviL}}} \phi & \iff \Gamma\VDash \phi \\
\Gamma \vdash_{U\textup{\textsc{EviL}$^\BM$}} \phi & \iff \Gamma\VDash \phi \\
\Gamma \vdash_{U\textup{\textsc{EviL}}^\BP} \phi & \iff \Gamma\VDash \phi
\end{align*}
\end{theorem}
\begin{proof}
Since by Theorem \ref{universal-completeness}, we know that 
$U\textup{\textsc{EviL}$^\BM$}$ and
  $U\textup{\textsc{EviL}$^\BP$}$ are subsystems of 
$U\textup{\textsc{EviL}}$, we need only prove that
\begin{align*}
\Gamma \vdash_{U\textup{\textsc{EviL}}} \phi & \iff \Gamma\VDash \phi 
\end{align*}
As usual, we only prove completeness.  Assume that $\Gamma
\vdash_{U\textup{\textsc{EviL}}} \phi$, then we know there is some
finite \textsc{EviL} Kripke structure $\mathbb{M}$ with a 
world $w$ such that
$\mathbb{M},w \nVdash \bigwedge \Gamma \to \phi$.  
Next, we may employ
induction to extend Lemma \ref{translation-lemma}, the \textsc{EviL}
translation lemma from \S\ref{translation}, to include among
other equivalences 
\[ \mathbb{M}, w \Vdash U \psi \iff \ipent, \kl(w) \VDash U \psi. \]
Here $U \psi $ is assumed to be a subformula of $\phi$.  This
establishes that $\ipent, \kl(w) \nVDash \bigwedge \Gamma \to \phi$,
giving the desired completeness result.
\end{proof}

As in the previous section, we admit that we have
made certain design choices here for the sake of simplicity.
Universal modality hints, however, at richer semantics one might
choose to develop.

For instance, we might imagine an our original concrete \textsc{EviL} models might
have, associated with them, a family of indexed accessibility
relations $R_X$ representing traditional epistemic logic 
 accessibility relations.
The semantics for $\Box_X \phi$ would then be characterized as:
\[ \Omega, (a,A) \VDash \Box_X \phi \iff \forall a R_X
b. \Omega,(b,B) \VDash A_X \textup{ implies } \Omega,(b,B) \VDash
\phi \]
It is straightforward to see that \textsc{EviL} is sound and complete
for these semantics, given our previous results.  We could then extend
\textsc{EviL} with traditional epistemic modalities corresponding to
the accessibility relations postulated.  Our original Theorem Theorem
would have to be relativized in the following manner:

\[ \Omega, (a,A) \VDash \Box_X \phi \iff Th(R[a]) \cup A \vdash \phi \]

Moreover, we could safely extend the grammar of the belief sets to include
formulae containing old-fashioned epistemic modalities, governed by
the accessibility relation. One could alternately investigate other extended
modalities as well, such as the \emph{difference} modality presented
in \cite[chapter 7.4]{van_benthem_modal_2010}.  Universal modality suggests that there
are many modifications that could potentially be made to \textsc{EviL}, 

However, a system where every agent is equipped with an accessibility
relation is more complicated than the simple, universal
modality semantics we have developed for \textsc{EviL}, 
As in \S\ref{subsystems}, we did not choose to modify
\textsc{EviL} in some of the more exotic ways one might imagine,
precisely because we wanted \textsc{EviL} to conform to our original
intuitions we developed in \S\ref{philosophy}.

%%% Local Variables: 
%%% mode: latex
%%% TeX-master: "evil_philosophy"
%%% End: 
