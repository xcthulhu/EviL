\subsubsection{Introduction}
In this section, we remark on the connection that single agent \textsc{EviL} bears
to \emph{intuitionistic logic}.

Informally, we may find inspiration in the following quote:
\begin{quote}
To be sure, intuitionistic logic, too, has its Kripke-style possible
world semantics \ldots Worlds stand for information states,
accessibility encodes possible informational growth, and truth at a
world corresponds intuitively to epistemic `forcing' by the available
evidence there.
\cite{van_benthem_reflectionsepistemic_1991}
\end{quote}

This key observation is analogous to our perspective on agents we took
up in \S\ref{quine}.  Recall that we thought of agents as
\emph{posets}, where higher nodes represented the agent embracing more 
evidence.  This can be regarded as the essentially the perspective as
the epistemic reading of intuitionistic logic given above.  On the
other hand, another \textsc{EviL} connection to traditional
intuitionistic logic may be seen found. This corresponds to a
well known proposal for constructive semantics extended by Hilary
Putnam, who suggests ``To claim a statement is true is to claim it
could be justified'' \cite{putnam_problem_1981}.  
Since our proposed The Theorem Theorem (Theorem
\ref{theorem-theorem} from \S\ref{evil-grammar}), equates $\Box$-boxed
formulae with justification, shall see that this philosophical insight
may be leveraged formal insight observation.

The rest of this section is devoted to illustrating the \textsc{EviL}
connection to intuitionistic logic.

\subsubsection{Preliminaries}

We first recall the Kripke semantics for intuitionistic logic may be given
as follows:

\begin{definition}[Intuitionistic Kripke Structures]
Let $\mathbb{P} = \langle W, \sqsubseteq, V\rangle$ be a Kripke
structure.  We say that $\mathbb{P}$ is \textbf{intuitionistic} if
$\sqsubseteq$ is transitive and reflexive and $V$ is \textbf{monotone}, that is
if $w \in V(p)$ and $v \sqsupseteq w$ then $v \in V(p)$ (here $v
\sqsupseteq w$ is just shorthand for $w \sqsubseteq v$).
\end{definition}

The of intuitionistic language is similar to the basic grammar
$\mathcal{L}_0$

\begin{definition}
Define the grammar $\mathcal{L}_I(\Phi)$ as follows:
\begin{eqnarray*}
& \phi\ {::=} \  p \in \Phi \ |\ \bot\ |\ \phi \to \psi \
|\ \phi \wedge \phi \ |\ \phi \vee \psi &\label{LI}
\end{eqnarray*}
\end{definition}

Likewise, the intuitionistic truth predicate $\Vvdash$ is similar to
$\Vdash$, and can be seen as a function which takes as input:
\begin{bul}
  \item A Kripke structure $\mathbb{P}$
  \item A world $w$
  \item An $\mathcal{L}_I$ formula $\phi$
\end{bul}
And outputs something is \textsf{bool}.  This may be written
technically as:
\[ (\Vvdash) \colons \mathcal{K}_{\Phi, I} \to I \to
\mathcal{L}_0(\Phi) \to \text{\textsf{bool}}\]
It is defined recursively as follows:
\begin{definition}
  Let $\mathbbm{P} = \langle W, \sqsubseteq, V\rangle$ be a Kripke structure:
  \begin{eqnarray*}
    \mathbbm{P}, w \Vvdash p & \Longleftrightarrow & w \in V
    (p)\\
    \mathbbm{P}, w \Vvdash \phi \rightarrow \psi & \Longleftrightarrow &
    \textup{for all $v \sqsupseteq w$: }\mathbbm{P}, v \Vvdash \phi \text{ implies } \mathbbm{P}, v \Vvdash \psi\\
    \mathbbm{P}, w \Vvdash \bot & \Longleftrightarrow & \text{False}\\
    \mathbbm{P}, w \Vvdash \phi \wedge \psi & \Longleftrightarrow & 
\mathbbm{P}, w \Vvdash \phi  \textup{ \& } \mathbbm{P}, w \Vvdash \psi
\\
    \mathbbm{P}, w \Vvdash \phi \vee \psi & \Longleftrightarrow & 
\mathbbm{P}, w \Vvdash \phi \textup{ or } \mathbbm{P}, w \Vvdash \psi
  \end{eqnarray*}
\end{definition}

Next, we present an axiomatization of \emph{Intuitionistic Logic},
which can be found in Table \ref{table:Iaxioms} below.  These axioms
are taken from \cite[chapter 5, pgs. 104--107]{urzyczyn_lecturescurry-howard_2006}.
\begin{table}
\centering
%\newcounter{rownum}
\setcounter{rownum}{0}
%\newcounter{rownum2}
\setcounter{rownum2}{0}
\begin{tabular}{|ll|}
\hline
  (\refstepcounter{rownum}U\arabic{rownum})&$ \vdash \phi \to \psi \to \phi$\\
   (\refstepcounter{rownum}U\arabic{rownum})&$ \vdash (\phi \to \psi
   \to \chi) \to (\phi \to \psi) \to \phi \to \chi$\\
   (\refstepcounter{rownum}U\arabic{rownum})&$ \vdash \phi \wedge \psi \to
   \phi$\\
 (\refstepcounter{rownum}U\arabic{rownum})&$ \vdash \phi \wedge \psi \to
 \psi$\\
 (\refstepcounter{rownum}U\arabic{rownum})&$ \vdash (\phi \to \psi)
 \to (\phi \to \chi) \to (\phi \to \psi \wedge \chi)$\\
 (\refstepcounter{rownum}U\arabic{rownum})&$ \vdash \phi \to \phi \vee \psi$\\
 (\refstepcounter{rownum}U\arabic{rownum})&$ \vdash \psi \to \phi \vee
 \psi$\\
 (\refstepcounter{rownum}U\arabic{rownum})&$ \vdash (\phi \to \chi)
 \to (\psi \to \chi) \to (\phi \vee \psi) \to \chi$\\
 (\refstepcounter{rownum}U\arabic{rownum})&$ \vdash \bot \to \phi$\\
(\refstepcounter{rownum2}\Roman{rownum2}) & \AxiomC{$\vdash \phi \to
  \psi$} \AxiomC{$\vdash \phi$} \BinaryInfC{$\vdash \psi$}
\DisplayProof \\ % & Modus Ponens\\%[10pt]
\hline
\end{tabular}
\caption{Intuitionistic Logic}
\label{table:Iaxioms}
\end{table}
Intuitionistic logic can be easily understood as the logic of
intuitionistic Kripke structures:
\begin{definition}
We shall write
\[ \Gamma \Vvdash_{Int} \phi \]
to mean that for all intuitionistic Kripke structures
$\mathbb{P} = \langle W, \sqsubseteq, V\rangle$,
for all worlds $w \in W$ if $\mathbb{P},w \Vvdash \Gamma$ then
$\mathbb{M},w \Vvdash \phi$.
\end{definition}
\begin{theorem}[Strong Intuitionistic Soundness and Completeness]\label{instroncomp}
\[ \Gamma \vdash_{Int} \phi \iff \Gamma \Vvdash_{Int} \phi \]
\end{theorem}
\begin{proof}
This is Proposition 5.1.10 from \cite[chapter 5, pg. 107]{urzyczyn_lecturescurry-howard_2006}.
\end{proof}

\subsubsection{The \textsc{EviL} G\"{o}del-Tarski-McKinsey Embedding}

In order understand how intutionistic logic connects to \textsc{EviL},
we shall first review the traditional G\"{o}del-Tarski-McKinsey
embedding of intuitionistic logic into the modal logic $S4$.  We
briefly review the grammar and axiomatics of $S4$ before proceeding:
\begin{definition}
Define the grammar $\mathcal{L}_{S4}(\Phi)$ as follows:
\begin{eqnarray*}
& \phi\ {::=} \  p \in \Phi \ |\ \bot\ |\ \phi \to \psi \
|\ \BP \phi &\label{LI}
\end{eqnarray*}
\end{definition}

The axiom systems $S4$ is listed in Table \ref{table:S4axioms}.  We
next review the completeness theorem for $S4$:
\begin{table}
\centering
%\newcounter{rownum}
\setcounter{rownum}{0}
%\newcounter{rownum2}
\setcounter{rownum2}{0}
\begin{tabular}{|ll|}
\hline
  (\refstepcounter{rownum}U\arabic{rownum})&$ \vdash \phi \to \psi \to \phi$\\
   (\refstepcounter{rownum}U\arabic{rownum})&$ \vdash (\phi \to \psi
   \to \chi) \to (\phi \to \psi) \to \phi \to \chi$\\
   (\refstepcounter{rownum}U\arabic{rownum})&
$ \vdash (\neg \psi \to \neg \phi) \to \phi \to \psi$\\
 (\refstepcounter{rownum}U\arabic{rownum})&$ \vdash \BP \phi \to \phi$\\
 (\refstepcounter{rownum}U\arabic{rownum})&$ \vdash \BP \phi \to \BP
 \BP \phi$\\
 (\refstepcounter{rownum}U\arabic{rownum})&$ \vdash \BP(\phi \to \psi)
 \to \BP \phi \to \BP \psi$\\
(\refstepcounter{rownum2}\Roman{rownum2}) & \AxiomC{$\vdash \phi \to
  \psi$} \AxiomC{$\vdash \phi$} \BinaryInfC{$\vdash \psi$}
\DisplayProof \\ % & Modus Ponens\\%[10pt]
(\addtocounter{rownum2}{1}\Roman{rownum2}) &
 $\AxiomC{$\vdash \phi$}
\UnaryInfC{$\vdash \BP \phi$}
\DisplayProof$  \\% [10pt]
\hline
\end{tabular}
\caption{The Modal Logic $S4$}
\label{table:S4axioms}
\end{table}

\begin{definition}
We shall write
\[ \Gamma \Vdash_{S4} \phi \]
to mean that for all Kripke structures
$\mathbb{P} = \langle W, \sqsubseteq, V\rangle$, where $\sqsubseteq$
is transitive and reflexive, for all worlds $w \in W$,
 if $\mathbb{P},w \Vdash \Gamma$ then
$\mathbb{P},w \Vdash \phi$.
\end{definition}

\begin{theorem}[$S4$ Strong Soundness and Completeness]
\[ \Gamma \vdash_{S4} \phi \iff \Vdash_{S4} \phi \]
\end{theorem}
\begin{proof}
This is Theorem 4.29 of \cite[chapter 4.3, pg. 205]{blackburn_modal_2001}.
\end{proof}

With the above, we may now provide the traditional G\"{o}del-Tarski-McKinsey embedding:

\begin{mydef}
The  G\"{o}del-Tarski-McKinsey embedding $(\cdot)^\circ :
\mathcal{L}_I \to \mathcal{L}_{S4}$ is a primitive recursively
defined function that takes formulae in the language of intuitionistic
logic to formulae in the language of $S4$.  It may given
programmatically as follows: 
\begin{align*}
  p^\circ & := \BP p \\
  \bot^\circ & := \bot \\
  (\phi \to \psi)^\circ & := \BP(\phi^\circ \to \psi^\circ) \\
  (\phi \wedge \psi)^\circ & := \phi^\circ \wedge \psi^\circ \\
  (\phi \vee \psi)^\circ & := \phi^\circ \vee \psi^\circ \\
\end{align*}
\end{mydef}

\begin{theorem}
\[ \Gamma \vdash_{Int} \phi \iff \Gamma^\circ \vdash_{S4} \phi^\circ \]
\end{theorem}
\begin{proof}
  This is Theorem 3.83 of \cite[chapter 3, pg. 97]{chagrov_modal_1997}.
\end{proof}

The \textsc{EviL} variation on the above embedding emanates from the
following observation, which holds for all \textsc{EviL} $\mathbb{M}$:

\[ \mathbb{M},w \Vdash \Box \phi\ \  \&\ \ w \subseteq v
\Longrightarrow \mathbb{M},w \Vdash \Box \phi \]

This is a consequence of the \textsc{EviL} property \ref{pV}.  Hence every \textsc{EviL} Kripke structure can be translated into an
intuitionistic structure in the following manner:

\begin{definition}
For every Kripke structure$\mathbb{M} = \langle W, R, \sqsubseteq,
\sqsupseteq, V, P \rangle$, define:
\[ \rho\mathbb{M} := \langle W, \sqsubseteq, V'\rangle\]
Where $V'(p) := \{ w \in W\ |\ \mathbb{M},v \Vdash \Box p$
\end{definition}

Since the Theorem Theorem asserts that we may interpret
$\mathfrak{M},(a,A) \VDash \Box p$ as ``the \textsc{EviL} agent has
can justify $p$ using her evidence $A$'', we may construe the above
definition as associating ``Truth'' with ``could be justified'',
following Hilary Putnam's suggestion in
\cite{putnam_problem_1981}. Note that this intuition emanates from the
special interpretation we gave concrete \textsc{EviL} models.  
On the other hand, while our intuitions are grounded in our concrete
semantics, we are unhindered by them.  All of our theorems take
place in the abstract semantics, where we may obtain our results in
a higher level of generality.

\begin{lemma}\labe{companion-int}
If $\mathbb{M}$ is \textsc{EviL} then $\rho \mathbb{M}$ is an intuitionistic Kripke structure.
\end{lemma}
\begin{proof}
As we previously remarked, $\rho\mathbb{M}$ is intuitionistic as a
consequence of the \textsc{EviL} property \ref{pV}.
\end{proof}

Thinking about this embedding, may arrive at our translation:

\begin{mydef}
The \textsc{EviL} G\"{o}del-Tarski-McKinsey embedding $(\cdot)^\skull:
\mathcal{L}_I(\Phi) \to \mathcal{L}_{\textup{\textsc{EviL}}}(\Phi)$ is a primitive recursively
defined function that takes formulae in the language of intuitionistic
logic to formulae in the language of $S4$.  It may given
programmatically as follows: 
\begin{align*}
  p^\skull & := \Box p \\
  \bot^\skull & := \bot \\
  (\phi \to \psi)^\skull & := \BP(\phi^\skull \to \psi^\skull) \\
  (\phi \wedge \psi)^\skull & := \phi^\skull \wedge \psi^\skull \\
  (\phi \vee \psi)^\skull & := \phi^\skull \vee \psi^\skull \\
\end{align*}
\end{mydef}

\begin{lemma}\label{companion}If $\mathbb{M}$ is an \textsc{EviL} Kripke structure,
  then for all worlds $w \in W$ and for all $\phi \in \mathcal{L}(\Phi)$:
\[ \rho\mathbb{M},w\Vvdash \phi \iff \Mathbb{M},w\Vdash \phi^\skull 
\] 
\end{lemma}
\begin{proof}
The proof proceeds by a trivial induction on $\phi$.
\end{proof}

Hence, every \textsc{EviL} Kripke structure may be coerced into an
intuitionistic Kripke structure which faithfully preserves the truth
of all intuitionistic formulae up to translation.

We may also observe that every intuitionistic Kripke structure may be
coerced into a \textsc{EviL} Kripke structure in a similar manner:

\begin{mydef}
For every Kripke structure $\mathbb{P} = \langle W, \sqsubseteq,
V\rangle$, and for every set of letters $\Phi$ define:
\[ \partial\mathbb{P} := \langle W', R, \preccurlyeq, \succcurlyeq,
V'\rangle, P\]
Where 
\begin{align*}
W' & := W \uplus \powerset \Phi \\
R & := \{(w,\Psi) \in W \times \powerset \Phi \ |\ \forall p \in
\Phi. \mathbb{P},w \Vvdash
p \Longrightarrow p \in \Psi \} \\
\preccurlyeq & :=\ \sqsubseteq \cup \{(\Psi,\Psi) \in \Phi \times \Phi\ |\ \Psi \subseteq
\Phi\} \\
\succcurlyeq & := \{(w,v) \in W\times W\ |\ v \preccurlyeq w\} \\
V'(p) & := \{ \Psi \subseteq \Phi\ | \ p \in \Psi\} \\
P & := \varnothing
\end{align*}
\end{mydef}

Before proceeding, the intuition behind the above construction is that
the we will leave the accessibility of the original intuitionistic
structure $\mathbb{P}$ intact, however we will commute local truth
valuations to non-local truth valuations by adding new worlds. 
The new worlds represent every possible extension of the truth values 
for one of the original intuitionistic structure.  
Moreover, each new world is an island unto itself.

The above coercion is enough to turn any intuitionist Kripke structure
into an \textsc{EviL} one:
\begin{lemma}\label{turningevil}
If $\mathbb{P}$ is an intuitionistic Kripke structure, then $\partial
\mathbb{P}$ is \textsc{EviL}
\end{lemma}
\begin{proof}
The \textsc{EviL} properties \ref{pI}, \ref{pII}, \ref{preverse} and
\ref{pVII} follow immediately by construction and the fact that
$\mathbb{P}$ is assumed to be intuitionistic. 
 \begin{description}
    % \item[\ref{pI}]
    % \item[\ref{pII}]
    % \item[\ref{preverse}]
    \item[\ref{islandiff}]  Assume that $w\preccurlyeq u$.  We must
      show that for all $p \in \Phi$, $w \in V(p) \iff u \in V(p)$.

If $w \in W^{\partial\mathbb{P}}$, then either $w \in W^\mathbb{P}$ or
$w \subseteq \Phi$.

      If $w \in W^\mathbb{P}$, then we know by construction that $w
      \preccurlyeq u \iff w \sqsubseteq^{\mathbb{P}} v$,  hence
       $u \in W^\mathbb{P}$.  We also know by construction that
       $\forall p \in \Phi. W^\mathbb{P} \cap V(p) = \varnothing$,
       hence we have the desired result.

     On the other hand, if $w \subseteq \Phi$ then by construction we
     have $w \preccurlyeq u \iff w = v$, which suffices.

    \item[\ref{pV}]  We must show $(R \circ \preccurlyeq) \subseteq
      R$.  So assume that $w \preccurlyeq v$ and $v R u$.  By
      construction it must be that 
         \begin{bul}
           \item $\{w,v\} \subseteq W^{\mathbb{P}}$ 
           \item There is some $\Psi$ where $u = \Psi \subseteq \Phi$
           \item If  $\mathbb{P},v \Vvdash p$ then $p \in \Psi$
         \end{bul}
      To show $w R u$ we must show $\mathbb{P},w \Vvdash p$ 
      then $p \in \Psi$.  So fix $p$ and 
       assume $\mathbb{P},w \Vvdash p$.  We 
      know by hypothesis that $\mathbb{P}$ is intuitionistic, hence 
        if $\mathbb{P},w \Vvdash p$
      then $\mathbb{P},v \Vvdash p$, since $w \sqsubseteq v$.  Hence
      $\mathbb{P},v \Vvdash p$.  However, we can conclude from above
      that $p \in \Psi$, which suffices.
    \item[\ref{pVI}]
We must show:
\begin{center}
 $(\preccurlyeq \circ R)
      \subseteq R$ \\
and\\
    $(\succurlyeq \circ R) \subseteq
    R$
\end{center}
Assume that $w R v$ and $u \sqsubseteq v$ or $v \sqsubseteq u$.  In
either case we must show $w R u$.  By construction
      it must be that $w \in W^\mathbb{P}$ and $v \subseteq \Phi$.  
However, we may
      again reason by construction that in either of the cases 
$u \sqsubseteq v$ or $v \sqsubseteq u$, we have $u = v$, whence $w R u$ as desired.
    \item[\ref{pVII}]  

 %   \item[]
  \end{description}
\end{proof}

\begin{lemma}\label{evilcompanion}
If $\mathbb{P}$ is an intuitionistic Kripke structure, then for all $w
\in W^{\mathbb{P}}$, we have: 
\[\mathbb{P},w\Vvdash \phi \iff \partial \mathbb{P},w \Vdash
\phi^\skull \]
\end{lemma}
\begin{proof}
This proceeds by routine induction on $\phi$.  The only case worth
mentioning is the case of $p \in \Phi$.

Assume that $\mathbb{P},w\Vvdash p$, then we know that if $w R \Psi$
then $p \in \Psi$, whence by construction $\partial \mathbb{P}, \Psi
\Vdash p$.  This means that $\partial \mathbb{P}, w\Vdash \Box p$, and
since $p^\skull = \Box p$, we have the desired result.

Next assume that $\partial \mathbb{P}, w\Vdash \Box p$, and let $\Xi
:= \{ q\in \Phi \ |\ \mathbb{P},w\Vvdash q\}$.  Evidently $w R \Xi$.
Moreover, by assumption we have that $\partial
\mathbb{P}, \Xi \Vdash p$.  By construction this implies that $\mathbb{P},
w\Vvdash p$, as desired.
\end{proof}

\begin{theorem}
\[ \Gamma \vdash_I \phi \iff \Gamma^\skull \vdash_{\textsc{EviL} \phi^\skull\]
\end{theorem}
\begin{proof}
$\Longrightarrow$: Assume that $\Gamma^\skull \nvdash_{\textsc{EviL}
  \phi^\skull$, we must show that $\Gamma \nvdash_{Int} \phi$.  By
  \textsc{EviL} completeness (Theorem \ref{evil-completeness}) we know
  there is some \textsc{EviL} Kripke structure $\mathbb{M}$ with a
  world $w$ such that
$\mathbb{M},w\Vdash \Gamma^\skull$ and
$\mathbb{M},w\nVdash \phi^\skull$.  By Lemma \ref{companion}, 
we know that $\rho\mathbb{M},w\Vvdash \Gamma$ and
$\rho\mathbb{M},w\nVvdash \phi$.  Since $\rho\mathbb{M}$ is
intuitionistic by Lemma \ref{companion-int}, and $Int$ is sound for
intuitionistic Kripke structures
(Theorem \ref{instroncomp}), we have that $\Gamma\nvdash_{Int}\phi$.

$\Longleftarrow$:  The proof proceeds as above, via contraposotion.
Difference is that here one uses
intuitionistic completeness (Theorem \ref{instroncomp}), Lemmas
\ref{turningevil} and \ref{evilcompanion} to coerce intuitionistic
structures to faithfully turn \textsc{EviL}, and finally \textsc{EviL}
soundness (Theorem \ref{evil-completeness}).
\end{proof}

%%% Local Variables: 
%%% mode: latex
%%% TeX-master: "evil_philosophy"
%%% End: 


