\subsubsection{Introduction}\label{intro-to-int}
In this section, we remark on the connection that single agent \textsc{EviL} bears
to \emph{intuitionistic logic}.

Informally, we may find inspiration in the following quote:
\begin{quote}
To be sure, intuitionistic logic, too, has its Kripke-style possible
world semantics \ldots Worlds stand for information states,
accessibility encodes possible informational growth, and truth at a
world corresponds intuitively to epistemic `forcing' by the available
evidence there.
\cite{van_benthem_reflectionsepistemic_1991}
\end{quote}

This key observation is analogous to our perspective on agents we took
up in \S\ref{quine}.  Recall that we thought of agents as
\emph{posets}, where higher nodes represented the agent embracing more 
evidence.  This can be regarded as the essentially the perspective as
the epistemic reading of intuitionistic logic given above.  On the
other hand, another \textsc{EviL} connection to traditional
intuitionistic logic may be seen found. This corresponds to a
well known proposal for constructive semantics extended by Hilary
Putnam, who suggests ``To claim a statement is true is to claim it
could be justified'' \cite{putnam_problem_1981}.  
Since our proposed Theorem Theorem (Theorem
\ref{theorem-theorem} from \S\ref{evil-grammar}) equates $\Box$-boxed
formulae with justification, we may leverage Putnam's philosophical insight
into a formal observation in \textsc{EviL}.

The rest of this section is devoted to illustrating the \textsc{EviL}
connection to intuitionistic logic.

\subsubsection{Preliminaries}

We first recall the Kripke semantics for intuitionistic logic may be given
as follows:

\begin{definition}[Intuitionistic Kripke Structures]
Let $\mathbb{P} = \langle W, \sqsubseteq, V\rangle$ be a Kripke
structure.  We say that $\mathbb{P}$ is \textbf{intuitionistic} if
$\sqsubseteq$ is transitive and reflexive and $V$ is \textbf{monotone}, that is
if $w \in V(p)$ and $v \sqsupseteq w$ then $v \in V(p)$ (here $v
\sqsupseteq w$ is just shorthand for $w \sqsubseteq v$).
\end{definition}

The of intuitionistic language is similar to the basic grammar
$\mathcal{L}_0(\Phi)$

\begin{definition}
Define the grammar $\mathcal{L}_{Int}(\Phi)$ as follows:
\begin{eqnarray*}
& \phi\ {::=} \  p \in \Phi \ |\ \bot\ |\ \phi \to \psi \
|\ \phi \wedge \phi \ |\ \phi \vee \psi &\label{LI}
\end{eqnarray*}
\end{definition}

Likewise, the intuitionistic truth predicate $\Vvdash$ is similar to
$\Vdash$, and can be seen as a function which takes as input:
\begin{bul}
  \item A Kripke structure $\mathbb{P}$
  \item A world $w$
  \item An $\mathcal{L}_{Int}(\Phi)$ formula $\phi$
\end{bul}
And outputs something is \textsf{bool}.  This may be written
technically as:
\[ (\Vvdash) \colons \mathcal{K}_{\Phi, I} \to I \to
\mathcal{L}_0(\Phi) \to \text{\textsf{bool}}\]
It is defined recursively as follows:
\begin{definition}
  Let $\mathbbm{P} = \langle W, \sqsubseteq, V\rangle$ be a Kripke structure:
  \begin{eqnarray*}
    \mathbbm{P}, w \Vvdash p & \Longleftrightarrow & w \in V
    (p)\\
    \mathbbm{P}, w \Vvdash \phi \rightarrow \psi & \Longleftrightarrow &
    \textup{for all $v \sqsupseteq w$: }\mathbbm{P}, v \Vvdash \phi \text{ implies } \mathbbm{P}, v \Vvdash \psi\\
    \mathbbm{P}, w \Vvdash \bot & \Longleftrightarrow & \text{False}\\
    \mathbbm{P}, w \Vvdash \phi \wedge \psi & \Longleftrightarrow & 
\mathbbm{P}, w \Vvdash \phi  \textup{ \& } \mathbbm{P}, w \Vvdash \psi
\\
    \mathbbm{P}, w \Vvdash \phi \vee \psi & \Longleftrightarrow & 
\mathbbm{P}, w \Vvdash \phi \textup{ or } \mathbbm{P}, w \Vvdash \psi
  \end{eqnarray*}
\end{definition}

Next, we present an axiomatization of \emph{Intuitionistic Logic},
which can be found in Table \ref{table:Iaxioms} below.  These axioms
are taken from \cite[chapter 5, pgs. 104--107]{urzyczyn_lecturescurry-howard_2006}.
\begin{table}
\centering
%\newcounter{rownum}
\setcounter{rownum}{0}
%\newcounter{rownum2}
\setcounter{rownum2}{0}
\begin{tabular}{|ll|}
\hline
  (\refstepcounter{rownum}U\arabic{rownum})&$ \vdash \phi \to \psi \to \phi$\\
   (\refstepcounter{rownum}U\arabic{rownum})&$ \vdash (\phi \to \psi
   \to \chi) \to (\phi \to \psi) \to \phi \to \chi$\\
   (\refstepcounter{rownum}U\arabic{rownum})&$ \vdash \phi \wedge \psi \to
   \phi$\\
 (\refstepcounter{rownum}U\arabic{rownum})&$ \vdash \phi \wedge \psi \to
 \psi$\\
 (\refstepcounter{rownum}U\arabic{rownum})&$ \vdash (\phi \to \psi)
 \to (\phi \to \chi) \to (\phi \to \psi \wedge \chi)$\\
 (\refstepcounter{rownum}U\arabic{rownum})&$ \vdash \phi \to \phi \vee \psi$\\
 (\refstepcounter{rownum}U\arabic{rownum})&$ \vdash \psi \to \phi \vee
 \psi$\\
 (\refstepcounter{rownum}U\arabic{rownum})&$ \vdash (\phi \to \chi)
 \to (\psi \to \chi) \to (\phi \vee \psi) \to \chi$\\
 (\refstepcounter{rownum}U\arabic{rownum})&$ \vdash \bot \to \phi$\\
(\refstepcounter{rownum2}\Roman{rownum2}) & \AxiomC{$\vdash \phi \to
  \psi$} \AxiomC{$\vdash \phi$} \BinaryInfC{$\vdash \psi$}
\DisplayProof \\ % & Modus Ponens\\%[10pt]
\hline
\end{tabular}
\caption{Intuitionistic Logic}
\label{table:Iaxioms}
\end{table}
Intuitionistic logic can be easily understood as the logic of
intuitionistic Kripke structures:
\begin{definition}
We shall write
\[ \Gamma \Vvdash_{Int} \phi \]
to mean that for all intuitionistic Kripke structures
$\mathbb{P} = \langle W, \sqsubseteq, V\rangle$,
for all worlds $w \in W$ if $\mathbb{P},w \Vvdash \Gamma$ then
$\mathbb{M},w \Vvdash \phi$.
\end{definition}
\begin{theorem}[Strong Intuitionistic Soundness and Completeness]\label{instroncomp}
\[ \Gamma \vdash_{Int} \phi \iff \Gamma \Vvdash_{Int} \phi \]
\end{theorem}
\begin{proof}
This is Proposition 5.1.10 from \cite[chapter 5, pg. 107]{urzyczyn_lecturescurry-howard_2006}.
\end{proof}

\subsubsection{The G\"{o}del-Tarski-McKinsey Embedding}

In order understand how intutionistic logic connects to \textsc{EviL},
we shall first review the traditional G\"{o}del-Tarski-McKinsey
embedding of intuitionistic logic into the modal logic $S4$.  We
briefly review the grammar and axiomatics of $S4$ before proceeding:
\begin{definition}
Define the grammar $\mathcal{L}_{S4}(\Phi)$ as follows:
\begin{eqnarray*}
& \phi\ {::=} \  p \in \Phi \ |\ \bot\ |\ \phi \to \psi \
|\ \BP \phi &
\end{eqnarray*}
\end{definition}

The axiom systems $S4$ is listed in Table \ref{table:S4axioms}.  We
next review the completeness theorem for $S4$:

\begin{table}
\centering
%\newcounter{rownum}
\setcounter{rownum}{0}
%\newcounter{rownum2}
\setcounter{rownum2}{0}
\begin{tabular}{|ll|}
\hline
  (\refstepcounter{rownum}U\arabic{rownum})&$ \vdash \phi \to \psi \to \phi$\\
   (\refstepcounter{rownum}U\arabic{rownum})&$ \vdash (\phi \to \psi
   \to \chi) \to (\phi \to \psi) \to \phi \to \chi$\\
   (\refstepcounter{rownum}U\arabic{rownum})&
$ \vdash (\neg \psi \to \neg \phi) \to \phi \to \psi$\\
 (\refstepcounter{rownum}U\arabic{rownum})&$ \vdash \BP \phi \to \phi$\\
 (\refstepcounter{rownum}U\arabic{rownum})&$ \vdash \BP \phi \to \BP
 \BP \phi$\\
 (\refstepcounter{rownum}U\arabic{rownum})&$ \vdash \BP(\phi \to \psi)
 \to \BP \phi \to \BP \psi$\\
(\refstepcounter{rownum2}\Roman{rownum2}) & \AxiomC{$\vdash \phi \to
  \psi$} \AxiomC{$\vdash \phi$} \BinaryInfC{$\vdash \psi$}
\DisplayProof \\ % & Modus Ponens\\%[10pt]
(\addtocounter{rownum2}{1}\Roman{rownum2}) &
 $\AxiomC{$\vdash \phi$}
\UnaryInfC{$\vdash \BP \phi$}
\DisplayProof$  \\% [10pt]
\hline
\end{tabular}
\caption{The Modal Logic $S4$}
\label{table:S4axioms}
\end{table}
\begin{definition}
We shall write
\[ \Gamma \Vdash_{S4} \phi \]
to mean that for all Kripke structures
$\mathbb{P} = \langle W, \sqsubseteq, V\rangle$, where $\sqsubseteq$
is transitive and reflexive, for all worlds $w \in W$,
 if $\mathbb{P},w \Vdash \Gamma$ then
$\mathbb{P},w \Vdash \phi$.
\end{definition}

\begin{theorem}[$S4$ Strong Soundness and Completeness]
\[ \Gamma \vdash_{S4} \phi \iff \Gamma \Vdash_{S4} \phi \]
\end{theorem}
\begin{proof}
This is Theorem 4.29 of \cite[chapter 4.3, pg. 205]{blackburn_modal_2001}.
\end{proof}

With the above, we may now provide the traditional
G\"{o}del-Tarski-McKinsey embedding, which establishes that
intuitionistic logic is a sublogic of $S4$ (up to translation):

\begin{mydef}[The G\"{o}del-Tarski-McKinsey Embedding]
The  \textbf{G\"{o}del-Tarski-McKinsey embedding} $(\cdot)^\circ :
\mathcal{L}_{Int}(\Phi) \to \mathcal{L}_{S4}(\Phi)$ is a recursively
defined function that takes formulae in the language of intuitionistic
logic to formulae in the language of $S4$.  It may given
programmatically as follows: 
\begin{align*}
  p^\circ & := \BP p \\
  \bot^\circ & := \bot \\
  (\phi \to \psi)^\circ & := \BP(\phi^\circ \to \psi^\circ) \\
  (\phi \wedge \psi)^\circ & := \phi^\circ \wedge \psi^\circ \\
  (\phi \vee \psi)^\circ & := \phi^\circ \vee \psi^\circ 
\end{align*}
\end{mydef}

\begin{theorem}\label{embedding}
\[ \Gamma \vdash_{Int} \phi \iff \Gamma^\circ \vdash_{S4} \phi^\circ \]
\end{theorem}
\begin{proof}
  This is Theorem 3.83 of \cite[chapter 3, pg. 97]{chagrov_modal_1997}.
\end{proof}

We now turn to providing an variation on the above
embedding.  This will allow us to observe a theorem similar Theorem
\ref{embedding}, only for
\textsc{EviL} instead of $S4$.  Our intuition for behind our novel
embedding begins with the 
following observation, which holds for all \textsc{EviL} $\mathbb{M}$:
\[ \mathbb{M},w \Vdash \Box \phi\ \  \&\ \ w \subseteq v
\Longrightarrow \mathbb{M},w \Vdash \Box \phi \]
This is a consequence of the \textsc{EviL} property \ref{pV}.  Hence every \textsc{EviL} Kripke structure can be translated into an
intuitionistic structure in the following manner:
\begin{definition}
For every Kripke structure $\mathbb{M} = \langle W, R, \sqsubseteq,
\sqsupseteq, V, P \rangle$, define:
\[ \rho\mathbb{M} := \langle W, \sqsubseteq, V'\rangle\]
Where $V'(p) := \{ w \in W\ |\ \mathbb{M},v \Vdash \Box p\}$
\end{definition}

Since the Theorem Theorem asserts that we may interpret
$\mathfrak{M},(a,A) \VDash \Box p$ as ``the \textsc{EviL} agent has
can justify $p$ using her evidence $A$'', we may construe the above
definition as associating ``Truth'' with ``could be justified'',
following Hilary Putnam's suggestion in
\cite{putnam_problem_1981} we previously discussed. Note that this intuition emanates from the
special interpretation we gave to concrete \textsc{EviL} models.  
On the other hand, while our intuitions are grounded in our concrete
semantics, we are unhindered by them.  All of our theorems take
place in the abstract semantics, where we may obtain our results in
a higher level of generality.

\begin{lemma}\label{companion-int}
If $\mathbb{M}$ is \textsc{EviL} then $\rho \mathbb{M}$ is an intuitionistic Kripke structure.
\end{lemma}
\begin{proof}
As we previously remarked, $\rho\mathbb{M}$ is intuitionistic as a
consequence of the \textsc{EviL} property \ref{pV}.
\end{proof}

Thinking about this embedding, may arrive at our translation, and
illustrate the connection it bears to our above translation in a lemma:

\begin{mydef}[The \textsc{EviL} G\"{o}del-Tarski-McKinsey Embedding]
The \textbf{\textsc{Evil} \textsc{EviL} G\"{o}del-Tarski-McKinsey embedding} $(\cdot)^\skull:
\mathcal{L}_{Int}(\Phi) \to \mathcal{L}_{\textup{\textsc{EviL}}}(\Phi)$ is a recursively
defined function that takes formulae in the language of intuitionistic
logic to formulae in the \textsc{EviL} language: 
\begin{align*}
  p^\skull & := \Box p \\
  \bot^\skull & := \bot \\
  (\phi \to \psi)^\skull & := \BP(\phi^\skull \to \psi^\skull) \\
  (\phi \wedge \psi)^\skull & := \phi^\skull \wedge \psi^\skull \\
  (\phi \vee \psi)^\skull & := \phi^\skull \vee \psi^\skull 
\end{align*}
\end{mydef}

\begin{lemma}\label{companion}
If $\mathbb{M}$ is an \textsc{EviL} Kripke structure,
  then for all worlds $w \in W$ and for all $\phi \in \mathcal{L}(\Phi)$:
\[ \rho\mathbb{M},w\Vvdash \phi \iff \mathbb{M},w\Vdash \phi^\skull 
\] 
\end{lemma}
\begin{proof}
The proof proceeds by a trivial induction on $\phi$.
\end{proof}

Hence, every \textsc{EviL} Kripke structure may be coerced into an
intuitionistic Kripke structure which faithfully preserves the truth
of all intuitionistic formulae up to translation.

We may also observe that every intuitionistic Kripke structure may be
coerced into a \textsc{EviL} Kripke structure in a similar manner:

\begin{mydef}[Diagonal Functor]
Define $$\Delta(S) := \{ (s,s) \in S \times S\ |\ s \in S\}.$$
This is known as the \textbf{Diagonal Functor} in the Category Theory
literature.  See \cite[chapter 9, pg. 181]{awodey_category_2006} for a
discussion of applications of this functor.
\end{mydef}

\begin{mydef}
For every Kripke structure $\mathbb{P} = \langle W, \sqsubseteq,
V\rangle$, and for every set of letters $\Phi$ define:
\[ \partial\mathbb{P} := \langle W', R, \preccurlyeq, \succcurlyeq,
V'\rangle, P\]
Where 
\begin{align*}
W' & := W \uplus \powerset \Phi \\
R & := \{(w,\Psi) \in W \times \powerset \Phi \ |\ \forall p \in
\Phi. \mathbb{P},w \Vvdash
p \Longrightarrow p \in \Psi \} \\
\preccurlyeq & :=\ \sqsubseteq \cup \Delta(\powerset \Phi) \\
\succcurlyeq & := \{(w,v) \in W\times W\ |\ v \preccurlyeq w\} \\
V'(p) & := \{ \Psi \subseteq \Phi\ | \ p \in \Psi\} \\
P & := \varnothing
\end{align*}
\end{mydef}

Before proceeding, the intuition behind the above construction is that
the we will leave the accessibility of the original intuitionistic
structure $\mathbb{P}$ intact, however we will commute local truth
valuations to non-local truth valuations by adding new worlds. 
The new worlds represent every possible extension of the truth values 
for one of the original intuitionistic structure.  
Moreover, each new world is an island unto itself.

The above coercion is enough to turn any intuitionist Kripke structure
into an \textsc{EviL} one:
\begin{lemma}\label{turningevil}
If $\mathbb{P}$ is an intuitionistic Kripke structure, then $\partial
\mathbb{P}$ is \textsc{EviL}
\end{lemma}
\begin{proof}
The \textsc{EviL} properties \ref{pI}, \ref{pII}, \ref{preverse} and
\ref{pVII} follow immediately by construction and the fact that
$\mathbb{P}$ is assumed to be intuitionistic. 
 \begin{description}
    % \item[\ref{pI}]
    % \item[\ref{pII}]
    % \item[\ref{preverse}]
    \item[\ref{pislandiff}]  Assume that $w\preccurlyeq u$.  We must
      show that for all $p \in \Phi$, $w \in V(p) \iff u \in V(p)$.

If $w \in W^{\partial\mathbb{P}}$, then either $w \in W^\mathbb{P}$ or
$w \subseteq \Phi$.

      If $w \in W^\mathbb{P}$, then we know by construction that $w
      \preccurlyeq u \iff w \sqsubseteq^{\mathbb{P}} v$,  hence
       $u \in W^\mathbb{P}$.  We also know by construction that
       $\forall p \in \Phi. W^\mathbb{P} \cap V(p) = \varnothing$,
       hence we have the desired result.

     On the other hand, if $w \subseteq \Phi$ then by construction we
     have $w \preccurlyeq u \iff w = v$, which suffices.

    \item[\ref{pV}]  We must show $(R \circ \preccurlyeq) \subseteq
      R$.  So assume that $w \preccurlyeq v$ and $v R u$.  By
      construction it must be that 
         \begin{bul}
           \item $\{w,v\} \subseteq W^{\mathbb{P}}$ 
           \item There is some $\Psi$ where $u = \Psi \subseteq \Phi$
           \item If  $\mathbb{P},v \Vvdash p$ then $p \in \Psi$
         \end{bul}
      To show $w R u$ we must show $\mathbb{P},w \Vvdash p$ 
      then $p \in \Psi$.  So fix $p$ and 
       assume $\mathbb{P},w \Vvdash p$.  We 
      know by hypothesis that $\mathbb{P}$ is intuitionistic, hence 
        if $\mathbb{P},w \Vvdash p$
      then $\mathbb{P},v \Vvdash p$, since $w \sqsubseteq v$.  Hence
      $\mathbb{P},v \Vvdash p$.  However, we can conclude from above
      that $p \in \Psi$, which suffices.
    \item[\ref{pVI}]
We must show:
\begin{center}
 $(\preccurlyeq \circ R)
      \subseteq R$ \\
and\\
    $ (\succcurlyeq \circ R) \subseteq
    R$
\end{center}
Assume that $w R v$ and $u \preccurlyeq v$ or $v \preccurlyeq u$.  In
either case we must show $w R u$.  By construction
      it must be that $w \in W^\mathbb{P}$ and $v \subseteq \Phi$.  
However, we may
      again reason by construction that in either of the cases 
$u \sqsubseteq v$ or $v \sqsubseteq u$, we have $u = v$, whence $w R u$ as desired.
   % \item[\ref{pVII}]  

 %   \item[]
  \end{description}
\end{proof}

\begin{lemma}\label{evilcompanion}
If $\mathbb{P}$ is an intuitionistic Kripke structure, then for all $w
\in W^{\mathbb{P}}$, we have: 
\[\mathbb{P},w\Vvdash \phi \iff \partial \mathbb{P},w \Vdash
\phi^\skull \]
\end{lemma}
\begin{proof}
This proceeds by routine induction on $\phi$.  The only case worth
mentioning is the case of $p \in \Phi$.

Assume that $\mathbb{P},w\Vvdash p$, then we know that if $w R \Psi$
then $p \in \Psi$, whence by construction $\partial \mathbb{P}, \Psi
\Vdash p$.  This means that $\partial \mathbb{P}, w\Vdash \Box p$, and
since $p^\skull = \Box p$, we have the desired result.

Next assume that $\partial \mathbb{P}, w\Vdash \Box p$, and let $\Xi
:= \{ q\in \Phi \ |\ \mathbb{P},w\Vvdash q\}$.  Evidently $w R \Xi$.
Moreover, by assumption we have that $\partial
\mathbb{P}, \Xi \Vdash p$.  By construction this implies that $\mathbb{P},
w\Vvdash p$, as desired.
\end{proof}

With the above established, we have enough to illustrate that
intuitionistic logic is a sublogic of \textsc{EviL}, after translation:

\begin{theorem}\label{evil_embedding}
\[ \Gamma \vdash_{Int} \phi \iff \Gamma^\skull \vdash_{\textsc{EviL}} \phi^\skull\]
\end{theorem}
\begin{proof}
$\Longrightarrow$: Assume that $\Gamma^\skull \nvdash_{\textsc{EviL}}
  \phi^\skull$, we must show that $\Gamma \nvdash_{Int} \phi$.  By
  \textsc{EviL} completeness (Theorem \ref{evil-completeness}) we know
  there is some \textsc{EviL} Kripke structure $\mathbb{M}$ with a
  world $w$ such that
$\mathbb{M},w\Vdash \Gamma^\skull$ and
$\mathbb{M},w\nVdash \phi^\skull$.  By Lemma \ref{companion}, 
we know that $\rho\mathbb{M},w\Vvdash \Gamma$ and
$\rho\mathbb{M},w\nVvdash \phi$.  Since $\rho\mathbb{M}$ is
intuitionistic by Lemma \ref{companion-int}, and $Int$ is sound for
intuitionistic Kripke structures
(Theorem \ref{instroncomp}), we have that $\Gamma\nvdash_{Int}\phi$.

$\Longleftarrow$:  The proof proceeds as above, via contraposition.
Difference is that here one uses
intuitionistic completeness (Theorem \ref{instroncomp}), Lemmas
\ref{turningevil} and \ref{evilcompanion} to coerce intuitionistic
structures to faithfully turn \textsc{EviL}, and finally \textsc{EviL}
soundness (Theorem \ref{evil-completeness}).
\end{proof}

\subsubsection{Van Benthem S4}

In this section we study how \textsc{EviL} relates to van Benthem
$S4$, which provides a simple abstraction of intuitionistic logic into
a modal setting.  Van Benthem $S4$ is a non-normal extension to $S4$
discussed in \cite{van_benthem_exploring_1996,van_benthem_information_2009,van_benthem_semantic_1989}.
It is axiomatized in Table \ref{vBS4axioms}.  It is essentially the
same as $S4$, only predicate letters are specified to be upward
monotone by a special, non-normal axiom.

\begin{table}
\centering
%\newcounter{rownum}
\setcounter{rownum}{0}
%\newcounter{rownum2}
\setcounter{rownum2}{0}
\begin{tabular}{|ll|}
\hline
  (\refstepcounter{rownum}U\arabic{rownum})&$ \vdash \phi \to \psi \to \phi$\\
   (\refstepcounter{rownum}U\arabic{rownum})&$ \vdash (\phi \to \psi
   \to \chi) \to (\phi \to \psi) \to \phi \to \chi$\\
   (\refstepcounter{rownum}U\arabic{rownum})&
$ \vdash (\neg \psi \to \neg \phi) \to \phi \to \psi$\\
 (\refstepcounter{rownum}U\arabic{rownum})&$ \vdash  p \to \BP p$\\
 (\refstepcounter{rownum}U\arabic{rownum})&$ \vdash \BP \phi \to \phi$\\
 (\refstepcounter{rownum}U\arabic{rownum})&$ \vdash \BP \phi \to \BP
 \BP \phi$\\
 (\refstepcounter{rownum}U\arabic{rownum})&$ \vdash \BP(\phi \to \psi)
 \to \BP \phi \to \BP \psi$\\
(\refstepcounter{rownum2}\Roman{rownum2}) & \AxiomC{$\vdash \phi \to
  \psi$} \AxiomC{$\vdash \phi$} \BinaryInfC{$\vdash \psi$}
\DisplayProof \\ % & Modus Ponens\\%[10pt]
(\addtocounter{rownum2}{1}\Roman{rownum2}) &
 $\AxiomC{$\vdash \phi$}
\UnaryInfC{$\vdash \BP \phi$}
\DisplayProof$  \\% [10pt]
\hline
\end{tabular}
\caption{Van Benthem $S4$}
\label{vBS4axioms}
\end{table}

Below, we offer several results that may be obtained for van Benthem
$S4$.  We do not intend to prove these theorems here; a curious reader
may read
\cite{van_benthem_exploring_1996,van_benthem_information_2009,van_benthem_semantic_1989}
for an in depth discussion of this logic.

\begin{definition}
We say that $\Gamma \vdash_{vBS4} \phi$ if and only if there is some
$\Sigma \subseteq_\omega \Gamma$ such that $\bigwedge \Sigma \to \phi$
is a theorem of van Benthem $S4$.
\end{definition}
\begin{proposition}[Van Benthem's  Strong Soundness and
  Completeness]\label{vBstrongsoundnesssandcompleteness}
$$\Gamma \Vdash_{Int} \phi \iff \Gamma \vdash_{vBS4} \phi$$
That is, van Benthem $S4$ is sound and strongly complete for
intuitionistic Kripke structures.
\end{proposition}

Note that the language of van Benthem $S4$ is modal, and
intuitionistic logic is not, hence the two logics are distinct.
Moreover, van Benthem $S4$ is a classical calculus, despite having
semantics over Intuitionistic Kripke structures.  The following
proposition summarizes the relationship of these calculi:

\begin{definition}[Van Benthem's Embedding]
\textbf{Van Benthem's embedding} $(\cdot)^\bullet :
\mathcal{L}_{Int}(\Phi) \to \mathcal{L}_{S4}(\Phi)$ is a recursively
defined function that takes formulae in the language of intuitionistic
logic to formulae in the language of $S4$: 
\begin{align*}
  p^\bullet & := p \\
  \bot^\bullet & := \bot \\
  (\phi \to \psi)^\bullet & := \BP(\phi^\bullet \to \psi^\bullet) \\
  (\phi \wedge \psi)^\bullet & := \phi^\bullet \wedge \psi^\bullet \\
  (\phi \vee \psi)^\bullet & := \phi^\bullet \vee \psi^\bullet 
\end{align*}
\end{definition}

Van Benthem's Embedding is essentially the same as the
G\"{o}del-Tarski-McKinsey embedding, save that predicate letters are
not boxed.  We may observe the following theorem:

\begin{proposition}[van Benthem's Embedding Theorem]\label{vBembedd}
\[ \Gamma \vdash_{Int} \phi \iff \Gamma^\bullet \vdash_{vBS4} \phi^\bullet \]
\end{proposition}

We now turn to illustrating how van Benthem's $S4$ is a proper
abstraction to think about \textsc{EviL} embeddings for intuitionistic
logic.  We may understand the embedding $\skull$ in terms of an
embedding of intuitionistic logic into van
Benthem $S4$:
\begin{definition}[Van Benthem's \textsc{EviL} Embedding]
\textbf{Van Benthem's \textsc{EviL} embedding} has the following type: $$(\cdot)^\Ankh :
\mathcal{L}_{S4}(\Phi) \to \mathcal{L}_{\textsc{EviL}}(\Phi)$$ It is a recursively
defined function that takes formulae in the language of $S4$ modal logic
logic to formulae in the \textsc{EviL} language: 
\begin{align*}
  p^\Ankh & := \Box p \\
  \bot^\Ankh & := \bot \\
  (\phi \to \psi)^\Ankh & := \phi^\Ankh \to \psi^\Ankh \\
(\BP \phi)^\Ankh & := \BP \phi^\Ankh
\end{align*}
\end{definition}
With the above we have two results:
\begin{proposition}
$$\phi^\skull = (\phi^\bullet)^\Ankh$$
\end{proposition}
\begin{lemma}\label{vBevilcompanions}
For any \textsc{EviL} Kripke structure $\mathbb{M}$:
$$\rho \mathbb{M},w \Vdash \phi \iff 
\mathbb{M},w \Vdash \phi^\Ankh$$
For any intuitionistic Kripke structure $\mathbb{P}$:
$$\mathbb{P},w \Vdash \phi \iff \partial \mathbb{P},w \Vdash \phi^\Ankh$$
\end{lemma}
\begin{proof}
In each case the proof follows from structural induction; they
essentially follow the proofs of Lemmas \ref{companion} and \ref{evilcompanion}.  
\end{proof}
Hence we may illustrate that van Benthem $S4$ is indeed a sublogic of
\textsc{EviL}, after translation:

\begin{theorem}\label{evil_embedding_2}
$$ \Gamma \vdash_{vBS4} \phi \iff \Gamma^\Ankh \vdash_{\textsc{EviL}}\phi^\Ankh $$
\end{theorem}
\begin{proof}
The proof follows the same structure as Theorem \ref{evil_embedding}.
Instead of appealing to Theorem \ref{instroncomp}, the strong
intuitionistic soundness and completeness, one uses Proposition
\ref{vBstrongsoundnesssandcompleteness}, van Benthem's
strong soundness and completeness theorem. 
Also, instead of using Lemmas \ref{companion} and \ref{evilcompanion},
one may simply appeal to Lemma \ref{vBevilcompanions}.
\end{proof}

The above theorems mean that \textsc{EviL} does not just embed
intuitionistic logic, but rather it may be elaborated to embed van
Benthem $S4$.  Hence, if one exhibits an embedding of van Benthem $S4$
into another system, one may automatically provide an embedding of
intuitionistic logic, since intuitionistic logic is essentially a
subcalculus of van Benthem $S4$.  In the subsequent sections, rather
than exhibiting how variations on the G\"{o}del-Tarski-McKinsey
embedding to illustrate the connection of intuitionistic logic to
\textsc{EviL}, we will instead focus on embedding van Benthem $S4$.
In every case, we can observe that simply composing the some embedding
of van Benthem $S4$ into \textsc{EviL} after an embedding of
intuitionistic logic into van Benthem $S4$ yields a novel embedding of
intuitionistic logic.  We are of the opinion that van Benthem $S4$
provides a good abstract domain to study the relationship between
intuitionistic structures and \textsc{EviL}, because of this phenomenon.

\subsubsection{Knowledge}

In this section, we present an alternative to the previous embedding of van Benthem
$S4$ (and hence intuitionstic logic) into \textsc{EviL}.  
This time, instead of associating truth conditions in 
intuitionistic logic with justifiability, we shall illustrate 
that they can be identified with
knowledge.  This illustrates that the remarks made of intuitionistic
logic taken from 
\cite{van_benthem_reflectionsepistemic_1991} that we quoted in \S\ref{intro-to-int} exhibits a clear interpretation
in the \textsc{EviL} semantics we have set forth.

\begin{definition}[The \textsc{EviL} Knowledge Embedding]
The \textbf{\textsc{EviL} knowledge embedding} $(\cdot)^\Biohazard :
\mathcal{L}_{S4}(\Phi) \to \mathcal{L}_{\textsc{EviL}}(\Phi)$ is a recursively
defined function that takes formulae in the language of $S4$ modal logic
logic to formulae in the \textsc{EviL} language: 
\begin{align*}
  p^\Biohazard & := K p \\
  \bot^\Biohazard & := \bot \\
  (\phi \to \psi)^\Biohazard & := \phi^\Biohazard \to \psi^\Biohazard \\
(\BP \phi)^\Biohazard & := \BP \phi^\Biohazard
\end{align*}
where $K p := \DM(\PP \wedge \Box p)$, as we first suggested in \S\ref{Descartes}
\end{definition}

The idea in the above embedding is to associate truth the notion of
knowledge presented previously in \S\ref{soundness}, namely that the
agent has a sound argument.  We may obtain the following result
regarding this embedding:

\begin{definition}
For every Kripke structure $\mathbb{M} = \langle W, R, \sqsubseteq,
\sqsupseteq, V, P \rangle$, define:
\[ \eta\mathbb{M} := \langle W, \sqsubseteq, V'\rangle\]
Where $V'(p) := \{ w \in W\ |\ \mathbb{M},v \Vdash K p\}$
\end{definition}

\begin{proposition}\label{companion2-int}
If $\mathbb{M}$ is \textsc{EviL} then $\eta \mathbb{M}$ is an intuitionistic Kripke structure.
\end{proposition}

Intuitively, the above lemma follows from the observation that, in the
concrete semantics, as the agent has
more propositions by which she may compose arguments, she has more
subsets of that basis to compose \emph{sound} arguments.  Since the
interpretation of knowledge for our definition of $K$ is precisely
the existence of a sound subsets of one's basic beliefs,  this
observation is codified in the following validity:
$$\vdash_{\textsc{EviL}}K\phi \to \BP K\phi$$ 
As in the previous embedding, we may also coerce intuitionistic
structures to become \textsc{EviL} in a manner appropriate for our embedding
$\Biohazard$, and obtain a correspondence lemma:

\begin{lemma}\label{bio-companion}
For all \textsc{EviL} Kripke structure $\mathbb{M}$, we have that:
\[ \eta \mathbb{M}, w \Vdash \phi \iff \mathbb{M}, w \Vdash
\phi^\Biohazard \]
\end{lemma}
\begin{proof}
This follows from a routine induction on the complexity of $\phi$.
\end{proof}

\begin{mydef}
For every Kripke structure $\mathbb{P} = \langle W, \sqsubseteq,
V\rangle$, and for every set of letters $\Phi$ define:
\[ \mu\mathbb{P} := \langle W', R, \preccurlyeq, \succcurlyeq,
V'\rangle, P\]
Where 
\begin{align*}
W' & := W \uplus \powerset \Phi \\
R & := W\times W \cup \{(w,\Psi) \in W \times \powerset \Phi \ |\ \forall p \in
\Phi. \mathbb{P},w \Vdash
p \Longrightarrow p \in \Psi \} \\
\preccurlyeq & :=\ \sqsubseteq \cup \Delta(\powerset \Phi) \\
\succcurlyeq & := \{(w,v) \in W\times W\ |\ v \preccurlyeq w\} \\
V'(p) & := W \cup \{ \Psi \subseteq \Phi\ | \ p \in \Psi\} \\
P & := W
\end{align*}
\end{mydef}

\begin{lemma}\label{turningevil3}
If $\mathbb{P}$ is an intuitionistic Kripke structure, then $\mu
\mathbb{P}$ is \textsc{EviL}
\end{lemma}
\begin{proof}
As in Lemma \ref{turningevil}, the \textsc{EviL} properties \ref{pI}, \ref{pII}, \ref{preverse} and
\ref{pVII} follow by construction, and the assumption that
$\mathbb{P}$ is intuitionistic. 
 \begin{description}
    % \item[\ref{pI}]
    % \item[\ref{pII}]
    % \item[\ref{preverse}]
    \item[\ref{pislandiff}]  Assume that $w\preccurlyeq u$.  We must
      show that for all $p \in \Phi$, $w \in V(p) \iff u \in V(p)$.

If $w \in W^{\partial\mathbb{P}}$, then either $w \in W^\mathbb{P}$ or
$w \subseteq \Phi$.

      If $w \in W^\mathbb{P}$, then we know by construction that $u
      \in W^\mathbb{P}$ as well.  In this case we have defined our
      structure such that for all $p \in \Phi$ we have $w \in V(p)$
      and $u \in V(p)$, so the result follows immediately.

     On the other hand, if $w \subseteq \Phi$ then it must be that $w \preccurlyeq u \iff w = v$, which suffices.

    \item[\ref{pV}]  We must show $(R \circ \preccurlyeq) \subseteq
      R$.  So assume that $w \preccurlyeq v$ and $v R u$.  We have by construction:
         \begin{bul}
           \item $\{w,v\} \subseteq W^{\mathbb{P}}$ 
           \item Either (A) $u \in W^{\mathbb{P}}$ or (B) there is some $\Psi$ where $u = \Psi \subseteq \Phi$
           \item If  $\mathbb{P},v \Vvdash p$ then $p \in \Psi$
         \end{bul}


      To show $w R u$ we have two cases for $u$.  If (A), then we know
      $u \in W^{mathbb{P}}$ so we have $w R u$ by construction. 
      
      On the other hand, in case (B) the argument is analogous to the
      proof given in Lemma \ref{turningevil}.
    \item[\ref{pVI}]
We must show:
\begin{center}
 $(\preccurlyeq \circ R)
      \subseteq R$ \\
and\\
    $(\succcurlyeq \circ R) \subseteq
    R$
\end{center}
Assume that $w R v$ and $u \preccurlyeq v$ or $v \preccurlyeq u$.  In
either case we must show $w R u$.  By construction
      it must be that $w \in W^\mathbb{P}$ and either (A) $v in
      W^\mathbb{P}$ or $v \subseteq \Phi$.  In case (A) it must
      be that $u \in W^\mathbb{P}$, hence $w R u$ by construction.
      In case (B), the argument again proceeds in a
      fashion analogous to Lemma \ref{turningevil}.

% However, we may
%       again reason by construction that in either of the cases 
% $u \sqsubseteq v$ or $v \sqsubseteq u$, we have $u = v$, whence $w R u$ as desired.
   % \item[\ref{pVII}]  

 %   \item[]
  \end{description}
\end{proof}

\begin{lemma}\label{evilcompanion3}
If $\mathbb{P}$ is an intuitionistic Kripke structure, then for all $w
\in W^{\mathbb{P}}$, we have: 
\[\mathbb{P},w\Vdash \phi \iff \mu \mathbb{P},w \Vdash
\phi^\Biohazard \]
\end{lemma}
\begin{proof}
This proceeds by routine induction on $\phi$.  As with Lemma
\ref{evilcompanion}, the only case worth
mentioning is the case of $p \in \Phi$.  

Assume that $\mathbb{P}, w \Vdash p$.  By construction we have that
$\mu \mathbb{P},w \Vdash \Box p$.  We can also easily see that
$\mu\mathbb{P},w\Vdash \PP$, hence $\mu \mathbb{P}, w \Vdash \PP \wedge
\Box p$.  Since $\preccurlyeq$ is reflexive, we have that $\mu \mathbb{P}, w \Vdash \DM(\PP \wedge
\Box p)$, which is to say that $\mu \mathbb{P}, w \Vdash K p$.

Now assume that $\mu \mathbb{P}, w \Vdash K p$, then there is some $v
\preccurlyeq u$ such that $\mu \mathbb{P}, v \Vdash \Box p$.  As in
the proof of Lemma
\ref{evilcompanion}, this implies that $\mathbb{P}, v \Vdash p$.
Since $\mathbb{P}$ is intuitionistic, we know that since $w
\succcurlyeq v$ we have $\mathbb{P}, w \Vdash p$, as desired.
\end{proof}

Hence, we may observe that $\Biohazard$ can be used to embed van
Benthem $S4$ into \textsc{EviL}.

\begin{theorem}\label{evil_embedding_3}
\[ \Gamma \vdash_{vBS4} \phi \iff \Gamma^\Biohazard \vdash_{\textsc{EviL}}\phi^\Biohazard\]
\end{theorem}
\begin{proof}
As in the proof of Theorems \ref{evil_embedding} and
\ref{evil_embedding_2}, employs existing completeness results
along with Proposition \ref{companion2-int} along with Lemmas
\ref{bio-companion}, \ref{turningevil3}, and \ref{evilcompanion3}.
\end{proof}

As a corollary, we have an embedding of intuitionistic logic
\begin{corollary}\label{inttoknowledge}
\[ \Gamma \vdash_{Int} \phi \iff (\Gamma^\bullet)^\Biohazard \vdash_{\textsc{EviL}}(\phi^\bullet)^\Biohazard\]
\end{corollary}
\begin{proof}
This follows immediately from Proposition \ref{vBembedd} and Theorem \ref{evil_embedding_3}. 
\end{proof}

\subsubsection{Imagination}

In this section we present one final embedding of intuitionistic logic
into \textsc{EviL}.  Previously, the inspiration behind the embeddings
was the idea that as one ascends in an intuitionistic
Kripke-structure, one has more evidence at their disposal to draw
conclusions.  This in turn meant that they could access fewer worlds,
since fewer worlds were compatible with their beliefs.

In this section, we illustrate that the \emph{dual} interpretation
also holds.  Now was one ``ascends'' in an intutionistic Kripke
structure, the agent holds \emph{fewer} beliefs, and considers more
things possible.

This gives rise to the following embedding:
\begin{mydef}[The \textsc{EviL} Imagination Embedding]
The \textbf{\textsc{EviL} imagination embedding} $(\cdot)^\radio:
\mathcal{L}_{Int}(\Phi) \to \mathcal{L}_{\textup{\textsc{EviL}}}(\Phi)$ is a recursively
defined function that takes formulae in the language of $S4$ modal logic
to formulae in the \textsc{EviL} language: 
\begin{align*}
  p^\radio & := \Pos p \\
  \bot^\radio & := \bot \\
  (\phi \to \psi)^\radio & := \phi^\radio \to \psi^\radio \\
  (\BP \phi)^\radio & := \BM \phi^\radio
\end{align*}
\end{mydef}

\begin{definition}
For every Kripke structure $\mathbb{M} = \langle W, R, \sqsubseteq,
\sqsupseteq, V, P \rangle$, define:
\[ \zeta\mathbb{M} := \langle W, \sqsupseteq, V'\rangle\]
Where $V'(p) := \{ w \in W\ |\ \mathbb{M},v \Vdash \Pos p\}$
\end{definition}

As in the previous cases, we may obtain the following results:

\begin{proposition}\label{imagin-companion-int}
If $\mathbb{M}$ is \textsc{EviL} then $\zeta \mathbb{M}$ is an intuitionistic Kripke structure.
\end{proposition}

As in the previous cases, the above proposition can be understood
naturally using the concrete semantics.  We know that if some
situation is compatible with an \textsc{EviL} agents beliefs, then if
that agent were to believe \emph{fewer} things, then that situation
would still be compatible with her beliefs.  This may be expressed as
the following validity:
\[ \vdash_{\textup{\textsc{EviL}}} \Pos \phi \to \BM \Pos \phi \]
This previous validity entails the above proposition.

\begin{lemma}\label{bio-companion2}
For all \textsc{EviL} Kripke structures $\mathbb{M}$
\[ \eta \mathbb{M}, w \Vdash \phi \iff \mathbb{M}, w \Vdash
\phi^\radio \]
\end{lemma}
\begin{proof}
As in the previous embeddings, this follows by induction on the complexity of $\phi$.
\end{proof}

\begin{mydef}
For every Kripke structure $\mathbb{P} = \langle W, \sqsubseteq,
V\rangle$, and for every set of letters $\Phi$ define:
\[ \xi\mathbb{P} := \langle W', R, \preccurlyeq, \succcurlyeq,
V'\rangle, P\]
Where 
\begin{align*}
W' & := W \uplus \powerset \Phi \\
R & := \{(w,\Psi) \in W \times \powerset \Phi \ |\ \forall p \in
\Phi.  p \in \Psi \Longrightarrow \mathbb{P},w \Vdash
p \} \\
\preccurlyeq & :=\ \sqsupseteq \cup \Delta(\powerset \Phi) \\
\succcurlyeq & := \{(w,v) \in W\times W\ |\ v \preccurlyeq w\} \\
V'(p) & := \{ \Psi \subseteq \Phi\ | \ p \in \Psi\} \\
P & := \varnothing
\end{align*}
\end{mydef}

It is worth stopping for a moment to discuss the nature of the
definition of $\xi$ and its relation to $\partial$.  

In $\partial \mathbb{P}$, we had that $w R \Psi$ if and only if $\mathbb{P}, w
\Vdash p$ for all $p \in \Psi$.  In this construction, a world $w$ in the 
original intuitionistic Kripke structure $\mathbb{P}$ can 
``see'' another a set of proposition 
letters $\Phi$ if and only if $\Psi$ extends the valuation at $w$.
Concisely, this is to say:
\begin{center}
$w R \Psi$ if and only if $\Psi$ is compatible with what is
\emph{believed} at $w$
\end{center}
``Future'' states from $w$ correspond to possible extensions to the
basic beliefs present at $w$.
Evidently, the intuition behind this construction is recognizably the similar 
as the intuition behind our original concrete semantics.

However, this is evidently not the only way to the atomic truth
conditions in intuitionistic logic; $\xi$ suggests an alternative perspective.  Instead of thinking of the atomic
letters that are true at a 
particular world $w$ in as assertions the agent believes, 
we invite the reader to interpret them as 
\emph{assertions the agent considers possible}.  Instead of the future
states of $w$ corresponding to informational extensions, the
correspond to ways in which an agent might open her mind greater
heights of imagination.  Under this reading, we will want to enforce
\begin{center}
$w R \Phi$ if and only if $\Phi$ is compatible with what is imagined
possible at $w$
\end{center}
This means that if $p \in \Psi$ but $\mathbb{P}, w \nVdash p$, then we
do not want $w R \Psi$.  On the other hand, if we have that for all $p
\in \Phi$ that if $\mathbb{P},w\nVdash p$ then $p \nin \Psi$, then
evidently $\Psi$ is not going to be making true any atomic formula
that is thought to be \emph{impossible} at $w$; the contrapositive of
this yields the necessary and sufficient conditions for $w R \Psi$
that we have given in $\xi$.

We may summarize the above as follows.  While the traditional reading of
intuitionistic Kripke structures is that they model agents acquiring
beliefs, or \emph{learning}, the reading behind the $\xi$ construction
is that they are can model \emph{forgetting} or Cartesian
\emph{doubting}, as we first suggested in \S\ref{Descartes}. 

\begin{lemma}\label{turningevil4}
If $\mathbb{P}$ is an intuitionistic Kripke structure, then $\xi
\mathbb{P}$ is \textsc{EviL}
\end{lemma}
\begin{proof}
As in the previous embedding theorems, we immediately have that properties \ref{pI}, \ref{pII}, \ref{preverse} and
\ref{pVII} hold.  The proofs of \ref{pislandiff} and \ref{pVI} follow the
arguments provided in the proof of Lemma \ref{turningevil}.
    
Hence, all we have left to show is \ref{pV}.  We must show $(R \circ \preccurlyeq) \subseteq
      R$.  So assume that $w \preccurlyeq v$ and $v R u$; we must show
      that $w R u$.  By construction there must be some $\Psi\subseteq
      \Phi$ such that $u = \Psi$, and moreover we have that $w
      \sqsupseteq^{\mathbb{P}} v$.  To show that $w R \Psi$ we must show
      that if $p \in \Psi$ then $\mathbb{P},w \Vdash p$.  But we know
      by construction that $v R \Psi$ implies that
      that if $p \in \Psi$ then $\mathbb{P},v \Vdash p$, and since
      intuitionistic Kripke structures are monotone and $w
      \sqsupseteq^{\mathbb{P}} v$, we have $\mathbb{P},w \Vdash p$ 
      as desired.
\end{proof}

This affords us our final embedding:

\begin{theorem}\label{evil_embedding_4}
\[ \Gamma \vdash_{vBS4} \phi \iff \Gamma^\radio \vdash_{\textsc{EviL}}\phi^\radio\]
\end{theorem}
\begin{proof}
As in the case of the previous embedding theorems, the above follows
from Proposition \ref{imagin-companion-int} Lemmas
\ref{bio-companion2}, \ref{turningevil4} and our previously established
completeness results.
\end{proof}

\begin{corollary}
\[ \Gamma \vdash_{Int} \phi \iff (\Gamma^\bullet)^\radio \vdash_{\textsc{EviL}}(\phi^\bullet)^\radio\]
\end{corollary}
\begin{proof}
As in the case of Corollary \ref{inttoknowledge}, this follows from Proposition \ref{vBembedd} and Theorem \ref{evil_embedding_4}. 
\end{proof}
This commences our investigations into the connection of \textsc{EviL}
to intuitionistic logic and intuitionistic Kripke structures.

%%% Local Variables: 
%%% mode: latex
%%% TeX-master: "evil_philosophy"
%%% End: 


