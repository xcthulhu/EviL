% \documentclass{article}

% \geometry{letterpaper}

% \begin{document}

\section{Introduction to Epistemic Logic}
\begin{frame}[allowframebreaks]
\frametitle{Introduction to Epistemic Logic}
\begin{itemizedot}
  \item Epistemic Logic (\tmtextsf{EL}) is a branch of applied modal logic
  
  \item Since its conception in Hintikka's , \tmtextsf{EL} traditionally takes
  knowledge as a primitive notion
  
  \item The language of basic \tmtextsf{EL} is the common language of modal
  logic:
  \[ \phi \hspace{1em} \colons = \hspace{1em} p \in \Phi \hspace{1em} |
     \hspace{1em} \phi \rightarrow \psi \hspace{1em} | \hspace{1em} \bot
     \hspace{1em} | \hspace{1em} K \phi \]
\end{itemizedot}
\framebreak
\begin{itemizedot}
  \item Modern \tmtextsf{EL} can be thought of as a general intentional
  framework for reasoning about information; that being the case it naturally
  finds applications beyond philosophy. \ Specifically, \tmtextsf{EL} finds
  application in:
  \begin{itemizedot}
    \item {\tmem{Computer Science}} -- in multi-agent systems and security
    protocol analysis
    
    \item \tmtextit{Economics} -- \tmtextsf{EL} is particularly suited to game
    theory; the notion of {\tmem{information states}} are shared between the
    two frameworks
  \end{itemizedot}
\end{itemizedot}
\end{frame}
\begin{frame}[allowframebreaks]
\frametitle{The Axioms of \tmtextsf{EL}}
\begin{itemizedot}
  \item Basic \tmtextsf{EL} of one agent is generally assumed to be the modal
  logic \tmtextsf{S5}
  
  \item The axioms of \tmtextsf{S5} are:
  \begin{itemizedot}
    \item All classical propositional tautologies
    
    \item $K \phi \rightarrow \phi${\hspace*{\fill}}Axiom $T$
    
    \tmtextit{Truth: If the agent knows something, then its true}
    
    \item $K \phi \rightarrow K K \phi${\hspace*{\fill}}Axiom $4$
    
    \tmtextit{Positive Introspection: If the agent knows something, she knows
    that she knows it}
    
    \item $\neg K \phi \rightarrow K \neg K \phi${\hspace*{\fill}}Axiom $5$
    
    \tmtextit{Negative Introspection: There are no unknown unknowns (despite
    Donald Rumsfeld)}
    
    \item $K (\phi \rightarrow \psi) \rightarrow K \phi \rightarrow K
    \psi${\hspace*{\fill}}Axiom $K$
    
    \tmtextit{Logical Omniscience: The agent knows all the logical
    consequences of her knowledge}
  \end{itemizedot}
\end{itemizedot}
\framebreak
\begin{itemizedot}
  \item \tmtextsf{EL} is also closed under the following rules:
  
  {\hspace*{\fill}}$\displaystyle \frac{\vdash \phi \hspace{1em} \vdash \phi \rightarrow
  \psi}{\vdash \psi}${\hspace*{\fill}}$\displaystyle \frac{\vdash \phi}{\vdash K
  \phi}${\hspace*{\fill}} \\
\end{itemizedot}
\end{frame}
\begin{frame}[allowframebreaks]
\frametitle{The Semantics of \tmtextsf{EL}}
\begin{itemizedot}
  \item The semantics of \tmtextsf{EL} are \tmtextit{Kripke Structures} with
  equivalence relations
  
  \item An equivalence relation $\sim$ over a set $W$ satisfies the following
  rules for all $x, y, z \in W$:
  \begin{itemizedot}
    \item $\sim$ is {\tmem{reflexive}} -- $x \sim x$
    
    \item $\sim$ is {\tmem{transitive}} -- if $x \sim y$ and $y \sim z$ then
    $x \sim z$
    
    \item $\sim$ is \tmtextit{symmetric} -- if $x \sim y$ then $y \sim x$
  \end{itemizedot}
\end{itemizedot}
\framebreak
\begin{itemizedot}
  \item Let $\mathbbm{M} \assign \langle W, \sim, V \rangle$ be a Kripke model
  where $\sim$ is an equivalence relation.
  
  \item The semantic truth predicate $(\Vdash)$ of \tmtextsf{EL} is defined as
  follows:
  \begin{itemizedot}
    \item $\mathbbm{M}, w \Vdash p \Longleftrightarrow p \in V (w)$
    
    \item $\mathbbm{M}, w \Vdash \phi \rightarrow \psi
    \Longleftrightarrow$either $\mathbbm{M}, w \nVdash \phi$ or $\mathbbm{M},
    w \Vdash \psi$
    
    \item $\mathbbm{M}, w \Vdash \bot$ never
    
    \item $\mathbbm{M}, w \Vdash K \phi \Longleftrightarrow$forall $v \in W$,
    if $w \sim v$ then $\mathbbm{M}, v \Vdash \phi$
  \end{itemizedot}
  \item Such a model is called an \tmtextsf{S5} model
\end{itemizedot}
\end{frame}

\begin{frame}
\frametitle{Thermometer in a Box}
\begin{itemizedot}
  \item Basic \tmtextsf{EL} ascribes knowledge to inanimate objects as readily
  as it models agents
  
  \item To understand this, consider the case of a thermometer in a $1$ m$^3$
  box:
  
  \begin{figure}[h]
\includegraphics[scale=.3]{../Humans/thermometer.pdf}
    \caption{Thermometer in a Box\label{thermometer-picture}}
  \end{figure}
 
  \item If the thermometer reads 290 Kelvin, how many moles of gas are in the
  chamber?
\end{itemizedot}
\end{frame}
\begin{frame}[allowframebreaks]
\frametitle{Ideal Gas Law}
\begin{itemizedot}
  \item There is no one answer for how many moles of gas are in the chamber. \
  Rather, the answer is governed by the \tmtextit{ideal gas law}:
  \[ P V = n R T \]
  \item Here $P$ is the \tmtextit{pressure}, $V$ is the \tmtextit{volume}, $n$
  stands for \tmtextit{moles}, $R$ is the \tmtextit{ideal gas constant}, and
  $T$ is the \tmtextit{temperature}
\end{itemizedot}
\end{frame}
\begin{frame}[allowframebreaks]
\frametitle{Thermometer Models}
\begin{itemizedot}
  \item We may use epistemic logic to model the beliefs of this thermometer in
  a box. \ The following minor adaptation of the basic epistemic logic grammar
  is suited to this purpose:
  \[ \phi \text{ } \colons = \text{ } x \text{ Pascals} \text{
     } | \text{ } y \text{ moles} \text{ } | \text{ } z \text{ Pascals} \text{
     } | \text{ } \phi \rightarrow \psi \text{ } | \text{ } \bot \text{ } |
     \text{ } K \phi \]
\end{itemizedot}
\framebreak
\begin{itemizedot}
  \item With appropriate interpretation we can think of an \tmtextsf{EL} model
  $\langle W, V, \sim \rangle$ for the thermometer in a box, expressed as
  follows. Remember that the volume in constant!
  \begin{itemizedot}
    \item $W$ is pairs $(P, n)$ where $P$ is some positive pressure in
    \tmtextit{Pascals} and $n$ is some number of moles
    
    \item $V$ is defined so:
    \begin{itemizedot}
      \item $(P, n) \in V (x \text{ Pascals})$ if $P = x$
      
      \item $(P, n) \in V (y \text{ moles})$ if $n = y$
      
      \item $(P, n) \in V (z \text{ Kelvin})$ if $z = \frac{P}{n \cdot R}$
    \end{itemizedot}
    \item Finally, $(P, n) \sim (P', n')$ if and only if $z = \frac{P}{n \cdot
    R}$
  \end{itemizedot}
\end{itemizedot}
\end{frame}
\begin{frame}
\frametitle{Thermometer Models (Visualization)}
\begin{itemizedot}
  \item We can visualize the information states of the thermometer in a box as
  follows:
  
  \begin{figure}[h]
    \includegraphics[scale=.5]{../Humans/therm.pdf} 
    \caption{Thermometer information states}
  \end{figure}
\end{itemizedot}
\end{frame}
\begin{frame}[allowframebreaks]
\frametitle{Thermometers are People Too $\ldots$ Sort of}
\begin{itemizedot}
  \item The view suggests that, under the view put forward by modern
  \tmtextsf{EL}, objects like thermometers have mental states just like
  people.
  
  \item This view originates with Daniel Dennet in his book \tmtextit{The
  Intentional Stance} (1989).  The principle difference is that Daniel
Dennet uses thermostats in his thinking rather than thermometers:
  
%   \begin{quote}
%     \tmtextit{It is not that we attribute (or should attribute) beliefs and
%     desires only to things in which we find internal representations, but
%     rather that when we discover some object for which the intentional
%     strategy works, we endeavor to interpret some of its internal states or
%     processes as internal representations. What makes some internal feature of
%     a thing a representation could only be its role in regulating the behavior
%     of an intentional system.}
% \end{quote}
    
%     \framebreak
\begin{quote}
    Now the reason for stressing our kinship with the thermostat should be
    clear. There is no magic moment in the transition from a simple thermostat
    to a system that really has an internal representation of the world around
    it. The thermostat has a minimally demanding representation of the world,
    fancier thermostats have more demanding representations of the world,
    fancier robots for helping around the house would have still more
    demanding representations of the world. Finally you reach
    us. (pg. 32)
  \end{quote}
\end{itemizedot}
\end{frame}
\begin{frame}
\frametitle{Unreasonable}
\begin{itemizedot}
  \item Sadly, while the above framework of \tmtextsf{EL} is a powerful system
  for analyzing informational aspects of knowledge, it does not do a very good
  job at modeling justifications
  
  \item \tmtextit{That is to say, \tmtextsf{S5} Kripke structures, like
  thermometers, can convey the informational interplay of knowledge for an
  epistemic agent, but they cannot adequately model anything recognizable as a
  \tmtextbf{reason} for possessing knowledge}
\end{itemizedot}
\end{frame}
\begin{frame}
\frametitle{Desiderata}
\begin{itemizedot}
  \item Roughly, one way model reasons in epistemic logic would be to enforce
  the following:
  
  \begin{center}
    $\mathbbm{M}, w \Vdash K \phi$ if and only if the agent has some kind of
    proof of $\phi$ at $w$
  \end{center}
  
  \item The purpose of \tmtextit{Evidentialist Logic}, or \tmtextsc{EviL}, is
  to provide a framework which enforces this
  
  \item Analogous approaches are taken by \tmtextit{Justification Logic},
  developed at CUNY, and in Fernando Velazquez-Quesada's PhD thesis
\end{itemizedot}
\end{frame}
\begin{frame}
\frametitle{The Language of \tmtextsc{EviL}}
\begin{itemizedot}
  \item Define the language $\mathcal{L}_0$ as basic propositional logic over
  an infinite set of proposition letters $\Phi$:
  \[ \phi \text{ } \colons = \text{ } p \in \Phi \text{ } | \text{ } \phi
     \rightarrow \psi \text{ } | \text{ } \bot \]
  \item The language $\mathcal{L}_{\text{\textsc{EviL}}}$ extends this as
  follows:
  \[ \phi \colons = \text{ } p \in \Phi \text{ } | \text{ } \phi \rightarrow
     \psi \text{ } | \text{ } \bot \text{ } | \text{ } \Box  \phi \text{ } |
     \text{ } \boxminus \phi \text{ } | \text{ } \boxplus \phi \text{ } |
     \text{ } \circlearrowleft \]
\end{itemizedot}
\end{frame}
\begin{frame}[allowframebreaks]
\frametitle{Reading the $\mathcal{L}_{\text{\textsc{EviL}}}$}
\begin{itemizedot}
  \item Before jumping into formal semantics, here is how to read these
  operations:
  \begin{itemizeminus}
    \item[$\Box  \phi$ ] means the agent can deduce $\phi$ from her basic beliefs
    
    \item[$\circlearrowleft$ ] means that all of the agent's basic beliefs are
    true
    
    \item[$\ominus \phi$ ] holds if there is a way the agent could cast some of
    her experience and background assumptions into doubt, and after which
    $\phi$ holds
    
    \item[$\oplus \phi$ ] holds if there is a way that the agent could extend her
    beliefs (or perhaps remember something) in some way, after which $\phi$
    holds
  \end{itemizeminus}
\end{itemizedot}
\framebreak
\begin{itemizedot}
  \item The following deserves special attention:
  \begin{itemizeminus}
    \item[$\Box  \phi$ ] means the agent can deduce $\phi$ from her basic beliefs
  \end{itemizeminus}
  \item This is in line with a \tmtextit{foundationalist} epistemology -
  namely that there are some beliefs that get the privilege of being
  {\tmem{basic}} or {\tmem{grounded}}. \ Other beliefs are then deduced
  appropriately from these basic beliefs.
  
  \item This is not so unreasonable -- to help develop one's intuitions
  regarding this, here are some examples of Basic and Non-Basic beliefs which
  some people might hold:
  \framebreak
   \begin{center}
     \begin{tabular}{|l|l|}
 \hline
       Basic & Non-Basic\\
\hline
 \hline
       I have a hand on the  & I cannot go to \\
 end of my arm & Albert Heijn right now \\
        & (because I'm in a Logic Tea talk)\\
 \hline
       Peano Arithmetic  & \tmtextsc{EviL} is consistent\\
  is consistent &  \\
 \hline
       & God is not benevolent\\
       God is benevolent & (because there is so \\
     &  much suffering in the world) \\
 \hline
     \end{tabular}
  \end{center}
\end{itemizedot}
\end{frame}
\begin{frame}
\frametitle{Charlotte}
\begin{itemizedot}
  \item To get a feel of how \tmtextsc{EviL} works, consider an agent named
  Charlotte. \ In this toy model, Charlotte has two basic beliefs:
  
  \begin{descriptiondash}
    \item[$\checked$] If Abelard has tried to kill Alex, then Alex has
    survived
    
    \item[$\checked$] Abelard has tried to kill Alex
  \end{descriptiondash}
  
  \item Assume these two statements are true. \ Then we have that
  \[ \circlearrowleft \wedge \Box  \text{``Alex has survived''} \]
  \item Moreover, logical conclusions drawn from true premises are true. \ In
  this little example, it is also the case that
  \[ \text{``Alex has survived''} \]
\end{itemizedot}
\end{frame}
\begin{frame}[allowframebreaks]
\frametitle{A First Attempt at a Definition of Knowledge}
\begin{itemizedot}
  \item One first idea of how to define knowledge in this system would then
  be:
  \[ K \phi \assign \circlearrowleft \wedge \Box  \phi \]
  \item This is not an adequate analysis, however
\end{itemizedot}
\begin{itemizedot}
  \item But suppose that Charlotte also believes the following:
  
  \begin{descriptiondash}
    \item[$\checked$] If Abelard has tried to kill Alex, then Alex has
    survived
    
    \item[$\checked$] Abelard has tried to kill Alex
    
    \item[$\times$] Vietnam is south of Malaysia
  \end{descriptiondash}
  
  \item The last statement is false$\ldots$
  
  \item Moreover, now not all of Charlotte's basic beliefs are true, so under
  the previous definition we have
  \[ \neg K \text{``Alex has survived''} \]
\end{itemizedot}
\framebreak

\begin{itemizedot}
  \item This is not in line with intuition however, since it seems like
  Vietnam's position with relation to Malaysia is {\tmem{irrelevant}} to
  conclusions Charlotte's might draw about Alex
  
  \item If Charlotte could {\tmem{cast into doubt}} the idea that ``Vietnam is
  south of Malaysia'', and then carry on deducing with only relevant
  information, she might reasonably be expected to have Knowledge
  
  \item A refined definition of knowledge that accomodates the above
  observation is:
  \[ K \phi \assign \ominus (\circlearrowleft \wedge \Box  \phi) \]
\end{itemizedot}
\end{frame}
\begin{frame}
\frametitle{Logical Omniscience}
\begin{itemizedot}
  \item This definition of knowledge has the feature that the following is
  \tmtextit{invalid} (known as logical omniscience):
  \[ \frac{K \phi \text{ \ \ \ } K (\phi \rightarrow \psi)}{K \psi}
  \times \]
\item To understand this, it is instructive to go over an example
\end{itemizedot}
\end{frame}
\begin{frame}[allowframebreaks]
\frametitle{Biologist}

  \begin{itemizeminus}
    \item[$A$ ] It's the harvest season for cranberries
    
    \item[$B$ ] If it's the harvest season for cranberries, there is a risk of
    bear attacks
    
    \item[$C$ ] A study in 2008 showed that black bear attacks in cranberry bogs
    in New England have been in steady decline
  \end{itemizeminus}

Pretend that a field biologist holds the beliefs above.

  She knows the first statement given her background in
  botony, and the second statement given her background in biology

\framebreak

  \begin{itemizeminus}
    \item[$\checked$ ] It's is the harvest season for cranberries
    
    \item[$\checked$ ] If it's the harvest season for cranberries, there is a
    risk of bear attacks
    
    \item[$\times$ ] A study in 2008 showed that black bear attacks in cranberry
    bogs in New England have been in steady decline
  \end{itemizeminus}
%\begin{itemize}
%\item  
Now assume that while the first two statements are true, the study
  is in fact erroneous -- perhaps not enough evidence was
  gathered
%\end{itemize}

\framebreak

  \begin{itemizeminus}
    \item[$A$ ] It's the harvest season for cranberries
    
    \item[$B$ ] If it's the harvest season for cranberries, bears will be
    attracted, hence there is a risk of bear attacks
    
    \item[$C$ ] A study in 2008 showed that black bear attacks in cranberry bogs
    in New England have been in steady decline
  \end{itemizeminus}
  And moreover, imagine that in the case of this biologist, she cannot think about bear attacks in
  cranberry bogs in New England without appealing to the above 2008 study

\vspace{.2cm}  

  That is just to say, when thinking about bears in New England, 
   she cannot cast the 2008 study into doubt and just use her other 
   beliefs as a biologist

\framebreak
\begin{itemizedot}
  \item The biologist will not have knowledge of the risk of bear attacks in
  this scenario, only inconsistent ideas on the subject
  
  \item Despite the fact that she knows that it is harvest season for
  cranberries and if it is harvest season for cranberries, there's a risk of
  bear attacks, she cannot formulate a sound argument for there being a risk
  of bear attacks
\end{itemizedot}
\end{frame}
\begin{frame}[allowframebreaks]
\frametitle{\tmtextsc{EviL} Semantics}
\begin{itemizedot}
  \item \tmtextsc{EviL} models are sets $\mathfrak{M} \subseteq \wp \Phi
  \times \wp \mathcal{L}_0$; that is to say they are sets of pairs $(a, A)$
  where:
  \begin{itemizedot}
    \item $a$ is a set of proposition letters
    
    \item $A$ is a set of proposition formulae -- the agent's basic beliefs
  \end{itemizedot}
\end{itemizedot}
\framebreak
\begin{itemizedot}
  \item The \tmtextsc{EviL} truth predicate $\Vvdash$ is defined recursively
  as follows:
  \begin{itemizedot}
    \item $\mathfrak{M}, (a, A) \Vvdash p \Longleftrightarrow p \in a$
    
    \item $\mathfrak{M}, (a, A) \Vvdash \phi \rightarrow \psi
    \Longleftrightarrow$either $\mathfrak{M}, (a, A) \not{\Vvdash} \phi$ or
    $\mathfrak{M}, (a, A) \Vvdash \psi$
    
    \item $\mathfrak{M}, (a, A) \Vvdash \bot$ never
    
    \item $\mathfrak{M}, (a, A) \Vvdash \Box  \phi \Longleftrightarrow \forall
    (b, B) \in \mathfrak{M} .$ if $\mathfrak{M}, (b, B) \Vvdash \alpha$ for
    all $\alpha \in A$, then $\mathfrak{M}, (b, B) \Vvdash \phi$
    
    \item $\mathfrak{M}, (a, A) \Vvdash \boxminus \phi \Longleftrightarrow
    \forall (b, B) \in \mathfrak{M} .$ if $a = b$ and $A \supseteq B$, then
    $\mathfrak{M}, (b, B) \Vvdash \phi$
    
    \item $\mathfrak{M}, (a, A) \Vvdash \boxplus \phi \Longleftrightarrow
    \forall (b, B) \in \mathfrak{M} .$ if $a = b$ and $A \subseteq B$, then
    $\mathfrak{M}, (b, B) \Vvdash \phi$
    
    \item $\mathfrak{M}, (a, A) \Vvdash \circlearrowleft
    \Longleftrightarrow$if $\mathfrak{M}, (a, A) \Vvdash \alpha$ for all
    $\alpha \in A$
  \end{itemizedot}
\end{itemizedot}
\end{frame}
\begin{frame}
\frametitle{Theorem Theorem}
\[ \mathfrak{M}, (a, A) \Vvdash \Box  \phi \Longleftrightarrow \tmop{Th} (
   \mathfrak{M}) \cup A \vdash_{\text{\tmtextsc{EviL}}} \phi \]
This illustrates that the semantics of \tmtextsc{EviL} reflect the intuitions
previously presented
\end{frame}


\begin{frame}
\frametitle{Conservative Extension}

Let $K$ denote basic modal logic. \ \tmtextsc{EviL} is a
\tmtextit{conservative extension} of $K$; that is, for any modal formula
$\phi$
\[ \vdash_K \phi \Longleftrightarrow \vdash_{\text{\tmtextsc{EviL}}} \phi \]
But $K$ is not the same as \tmtextsc{EviL}, since in modal logic you do not
have $\boxplus, \boxminus$ and $\circlearrowleft$
\end{frame}


\begin{frame}
\frametitle{\tmtextsc{EviL} Is Not Compact}

Suppose the set of proposition letters $\Phi$ is infinite. \ Consider the
infinite set of formulae, $\tau [\Phi]$, where $\tau : \Phi \rightarrow
\mathcal{L}_{\text{\tmtextsc{EviL}}}$ is:
\[ \tau (p) \assign p \wedge \Box  p \wedge \diamondsuit \top \wedge \Box \Box 
   \bot \]
Every finite subset of $\tau [\Phi]$ is satisfiable in \tmtextsc{EviL}, but
not the entirety

\vspace{.2cm}

Hence, \tmtextsc{EviL} is \tmtextit{not compact}
\end{frame}


\begin{frame}[allowframebreaks]
\frametitle{\tmtextsc{EviL} Axioms}

\textsc{EviL} is \tmtextit{not normal}; it is not closed under replacement
of proposition letters with arbitrary formulae. \ Hence the axioms of
\textsc{EviL} are schematic:

\begin{center}
  \begin{tabular}{ll}
    $\vdash_{\text{\tmtextsc{EviL}}} \phi \rightarrow \psi \rightarrow \phi$ &
    $\vdash_{\text{\textsc{EviL}}} p \rightarrow \boxplus p \hspace{1em}
    (\ast)$\\
    $\vdash_{\text{\tmtextsc{EviL}}} (\phi \rightarrow \psi \rightarrow \chi)
    \rightarrow (\phi \rightarrow \psi) \rightarrow \phi \rightarrow \chi$ &
    $\vdash_{\text{\textsc{EviL}}} p \rightarrow \boxminus p \hspace{1em}
    (\ast)$\\
    $\vdash_{\text{\tmtextsc{EviL}}} (\neg \phi \rightarrow \neg \psi)
    \rightarrow \psi \rightarrow \phi$ & $\vdash_{\text{\textsc{EviL}}} \phi
    \rightarrow \boxminus \oplus \phi$\\
    $\vdash_{\text{\tmtextsc{EviL}}} \Box(\phi \rightarrow \psi) \rightarrow
    \Box \phi \rightarrow \Box \psi$ & $\vdash_{\text{\textsc{EviL}}} \phi
    \rightarrow \boxplus \ominus \phi$\\
    $\vdash_{\text{\tmtextsc{EviL}}} \boxminus (\phi \rightarrow \psi)
    \rightarrow \boxminus \phi \rightarrow \boxminus \psi$ &
    $\vdash_{\text{\textsc{EviL}}} \Box \phi \rightarrow \Box \boxminus
    \phi$\\
    $\vdash_{\text{\tmtextsc{EviL}}} \boxplus (\phi \rightarrow \psi)
    \rightarrow \boxplus \phi \rightarrow \boxplus \psi$ &
    $\vdash_{\text{\textsc{EviL}}} \Box \phi \rightarrow \Box \boxplus \phi$
  \end{tabular}

\end{center}
\framebreak
\begin{center}
\begin{tabular}{l}
  $\vdash_{\text{\textsc{EviL}}} \boxminus \phi \rightarrow \phi$\\
  $\vdash_{\text{\textsc{EviL}}} \boxminus \phi \rightarrow \boxminus
  \boxminus \phi$\\
  $\vdash_{\text{\textsc{EviL}}} \Box \phi \rightarrow \boxplus \Box \phi$\\
  $\vdash_{\text{\tmtextsc{EviL}}} \circlearrowleft \rightarrow \Box \phi
  \rightarrow \phi$\\
  $\vdash_{\text{\tmtextsc{EviL}}} \circlearrowleft \rightarrow \boxminus
  \circlearrowleft$
\end{tabular}

\end{center}

\framebreak
$(\ast)$ indicates a non-normal axiom

\textsc{EviL} is closed under modus ponens, and Necessitation for $\Box $,
$\boxminus$ and $\boxplus$
\end{frame}

\begin{frame}
\frametitle{\textsc{EviL} Soundness and Weak Completeness}

For all formulae $\phi$ in $\mathcal{L}_{\text{\tmtextsc{EviL}}}$
\begin{eqnarray*}
  & \mathfrak{M,} (a, A) \Vvdash \phi \text{ for all \tmtextsc{EviL} models
  $\mathfrak{M}$, for all $(a, A) \in \mathfrak{M}$} & \\
  & \text{if and only if} & \\
  & \vdash_{\text{\tmtextsc{EviL}}} \phi & 
\end{eqnarray*}
Moreover, \tmtextsc{EviL} has the finite model property (where all of the
basic beliefs sets are finite too)
\end{frame}


\begin{frame}
\frametitle{Complexity}

\tmtextsc{EviL} is decidable; specifically, we have the following bounds on
its complexity:

\begin{center}
  PSPACE$\subseteq$\tmtextsc{EviL}$\subseteq$EXP2
\end{center}
\end{frame}


\begin{frame}[allowframebreaks]
\frametitle{Elimination Theorem}

One way to read $\Box  \phi$ as ``the agent believes $\phi$'' and $\diamondsuit
\phi$ as ``the agent can imagine $\phi$''

\vspace{.2cm}

It is well known that they are each other's \tmtextit{dual}, so they exhibit
quite a bit of symmetry

\vspace{.2cm}

Since $\boxplus$ is associated with going up in basic beliefs, it may be
thought of as sort of symmetrical to $\boxminus$, which is associated with
going down in basic beliefs

\framebreak

This symmetry is exhibited in the following validities:

\begin{center}
  \begin{tabular}{ll}
    $\vdash_{\text{\tmtextsc{EviL}}} \boxminus p \leftrightarrow p$ &
    $\vdash_{\text{\tmtextsc{EviL}}} \boxplus p \leftrightarrow p$\\
    $\vdash_{\text{\tmtextsc{EviL}}} \boxminus \neg p \leftrightarrow \neg p$
    & $\vdash_{\text{\tmtextsc{EviL}}} \boxplus \neg p \leftrightarrow \neg
    p$\\
    $\vdash_{\text{\tmtextsc{EviL}}} \boxminus \diamondsuit \phi
    \leftrightarrow \diamondsuit \phi$ & $\vdash_{\text{\tmtextsc{EviL}}}
    \boxplus \Box  \phi \leftrightarrow \Box  \phi$\\
    $\vdash_{\text{\tmtextsc{EviL}}} \oplus \diamondsuit \phi \leftrightarrow
    \diamondsuit \phi$ & $\vdash_{\text{\tmtextsc{EviL}}} \ominus \Box  \phi
    \leftrightarrow \Box  \phi$\\
    $\vdash_{\text{\tmtextsc{EviL}}} \boxminus \boxminus \phi \leftrightarrow
    \boxminus \phi$ & $\vdash_{\text{\tmtextsc{EviL}}} \boxplus \boxplus \phi
    \leftrightarrow \boxplus \phi$\\
    $\vdash_{\text{\tmtextsc{EviL}}} \boxminus \oplus \phi \leftrightarrow
    \oplus \phi$ & $\vdash_{\text{\tmtextsc{EviL}}} \boxplus \ominus \phi
    \leftrightarrow \ominus \phi$\\
    $\vdash_{\text{\tmtextsc{EviL}}} \boxminus \circlearrowleft
    \leftrightarrow \circlearrowleft$ & $\vdash_{\text{\tmtextsc{EviL}}}
    \boxplus \neg \circlearrowleft \leftrightarrow \neg \circlearrowleft$
  \end{tabular}
\end{center}

Note that all of these validities are \tmtextit{reductions} -- the formula on
left of the biimplication is always more complex than the formula on the right

\framebreak

Now consider the following two \tmtextit{dual} fragments of
$\mathcal{L}_{\text{\tmtextsc{EviL}}}$:

{\hspace*{\fill}}$\phi \text{ } \colons = \text{ } p \text{ } | \text{ } \neg
p \text{ } | \text{ } \top \text{ } | \text{ } \bot \text{ } | \text{ }
\circlearrowleft \text{ } | \text{ } \phi \vee \psi \text{ } | \text{ } \phi
\wedge \psi \text{ } | \text{ } \diamondsuit \phi \text{ } | \text{ }
\boxminus \phi \text{ } | \text{ } \oplus
\phi${\hspace*{\fill}}($\mathcal{L}_A$)

\vspace{.2cm}

{\hspace*{\fill}}$\phi \text{ } \colons = \text{ } \neg p \text{ } | \text{ }
p \text{ } | \text{ } \bot \text{ } | \text{ } \top \text{ } | \text{ } \neg
\circlearrowleft \text{ } | \text{ } \phi \wedge \psi \text{ } | \text{ } \phi
\vee \psi \text{ } | \text{ } \Box  \phi \text{ } | \text{ } \ominus \phi
\text{ } | \text{ } \boxplus \phi${\hspace*{\fill}}($\mathcal{L}_B$)

\vspace{.2cm}

The observed reductions give rise to \tmtextit{elimination theorems} on these
fragments

\framebreak

Specifically, recursively define an \tmtextit{elimination operation}
$(\cdot)^{\ast}$ such that

\begin{center}
  \begin{tabular}{ll}
    $p^{\ast} \assign p$ & $(\neg p)^{\ast} \assign \neg p$\\
    $\top^{\ast} \assign \top$ & $\bot^{\ast} \assign \bot$\\
    $\circlearrowleft^{\ast} \assign \circlearrowleft$ & $(\neg
    \circlearrowleft)^{\ast} \assign \neg \circlearrowleft$\\
    $(\phi \vee \psi)^{\ast} \assign \phi^{\ast} \vee \psi^{\ast}$ & $(\phi
    \wedge \psi)^{\ast} \assign \phi^{\ast} \wedge \psi^{\ast}$\\
    $(\diamondsuit \phi)^{\ast} \assign \diamondsuit (\phi^{\ast})$ & $(\Box 
    \phi)^{\ast} \assign \Box (\phi^{\ast})$\\
    $(\boxminus \phi)^{\ast} \assign \phi^{\ast}$ & $(\ominus \phi)^{\ast}
    \assign \phi^{\ast}$\\
    $(\oplus \phi)^{\ast} \assign \phi^{\ast}$ & $(\boxplus \phi)^{\ast}
    \assign \phi^{\ast}$
  \end{tabular}
\end{center}

\framebreak

For $\phi \in \mathcal{L}_A \cup \mathcal{L}_B$ we have
\[ \vdash_{\text{\tmtextsc{EviL}}} \phi \leftrightarrow \phi^{\ast} \]
This is a consequence of the previous observed reductions, along with duality

\framebreak

This gives rise to the following \tmtextsc{EviL} validity

\newcommand{\DP}{\MNSdiamondminus}
\newcommand{\BM}{\MNSboxplus}
\begin{eqnarray*}
\vdash_{\text{\textsc{EviL}}} & \DP\DP\DP\DP\DP\DP\DP\DP\DP\DP\DP\DP\DP\DP\DP\DP\DP\DP\ \\ 
& \DP\DP\DP\DP\DP\BM\BM\BM\BM\BM\BM\BM\BM\BM\BM\DP\DP\DP\DP\DP\ \\ 
& \DP\DP\DP\DP\BM\BM\Box\Box\Box\Box\Box\Box\Box\Box\BM\BM\DP\DP\DP\DP\ \\ 
& \DP\DP\BM\BM\Box\Box\Box\Box\Box\Box\Box\Box\Box\Box\Box\Box\BM\BM\BM\DP\ \\ 
& \DP\BM\Box\Box\Box\Box\DP\DP\Box\Box\Box\Box\DP\DP\Box\Box\Box\Box\BM\DP\ \\ 
& \DP\BM\Box\Box\Box\Box\DP\DP\Box\Box\Box\Box\DP\DP\Box\Box\Box\Box\BM\DP\ \\ 
& \DP\BM\Box\Box\Box\Box\Box\Box\Box\Box\Box\Box\Box\Box\Box\Box\Box\Box\BM\DP\ \\ 
& \DP\BM\Box\Box\Box\DP\Box\Box\Box\Box\Box\Box\Box\Box\DP\Box\Box\Box\BM\DP\ \\ 
& \DP\BM\BM\Box\Box\DP\DP\DP\DP\DP\DP\DP\DP\Box\Box\BM\BM\DP\ \\ 
&
\DP\DP\DP\BM\BM\Box\Box\Box\Box\Box\Box\Box\Box\Box\Box\BM\BM\DP\DP\DP\
\\ 
& \DP\DP\DP\DP\BM\BM\BM\BM\BM\BM\BM\BM\BM\BM\BM\BM\DP\DP\DP\DP\  \\ 
& \DP\DP\DP\DP\DP\DP\DP\DP\DP\DP\DP\DP\DP\DP\DP\DP\DP\top & 
\end{eqnarray*}
\end{frame}



\begin{frame}[allowframebreaks]
\frametitle{\tmtextsc{EviL} Intuitionism}

In \tmtextit{Reason, Truth and History} (1981), Hilary Putnam writes:

\begin{quote}
  To claim a statement is true is to claim that it could be justified (pg. 56)
\end{quote}

This intuition presents two things:
\begin{enumeratenumeric}
  \item It suggests an anti-realist, constructivist perspective on truth
  
  \item Reading $\Box  p$ as ``$p$ can be justified'' is natural in
  \tmtextsc{EviL}. This suggests that truth conditions in models for
  intuitionistic logic can be translated into deductions an \tmtextsc{EviL}
  agent might perform
\end{enumeratenumeric}
\framebreak

Recall the grammar of intuitionistic logic:

{\hspace*{\fill}}$\phi \text{ } \colons = \text{ } p \text{ } | \text{ } \bot
\text{ } | \text{ } \phi \vee \psi \text{ } | \text{ } \phi \wedge \psi \text{
} | \text{ } \phi \rightarrow
\psi${\hspace*{\fill}}($\mathcal{L}_{\tmop{Int}}$)

\vspace{.2cm}

Define $(\cdot)^{\text{\tmtextsc{EviL}}} : \text{$\mathcal{L}_{\tmop{Int}}$}
\rightarrow \mathcal{L}_{\text{\tmtextsc{EviL}}}$ to be a variation on the
G\"{o}del Tarski McKinsey embedding:

\begin{center}
  \begin{tabular}{ll}
    $p^{\text{\tmtextsc{EviL}}} \assign \Box p$ &
    $\bot^{\text{\tmtextsc{EviL}}} \assign \bot$\\
    $(\phi \vee \psi)^{\text{\tmtextsc{EviL}}} \assign
    (\phi^{\text{\tmtextsc{EviL}}} \vee \psi^{\text{\tmtextsc{EviL}}})^{}$ &
    $(\phi \wedge \psi)^{\text{\tmtextsc{EviL}}} \assign
    (\phi^{\text{\tmtextsc{EviL}}} \wedge \psi^{\text{\tmtextsc{EviL}}})^{}$\\
    $(\phi \rightarrow \psi)^{\text{\tmtextsc{EviL}}} \assign \boxplus
    (\phi^{\text{\tmtextsc{EviL}}} \rightarrow \psi^{\text{\tmtextsc{EviL}}})$
    & 
  \end{tabular}
\end{center}

\framebreak

We have the following:
\[ \vdash_{\tmop{Int}} \phi \Longleftrightarrow
   \vdash_{\text{\tmtextsc{EviL}}} \phi^{\text{\tmtextsc{EviL}}} \]
Intuitionist Logic as \tmtextsc{EviL} Epistemic Logic

\vspace{.2cm}

The above embedding involves equating truth conditions in Intuitionistic
Kripke Structures with beliefs in \tmtextsc{EviL} models

\vspace{.2cm}

It may be modified to $(\cdot)^{\text{\tmtextsc{EviiL}}} :
\text{$\mathcal{L}_{\tmop{Int}}$} \rightarrow
\mathcal{L}_{\text{\tmtextsc{EviL}}}$

\vspace{.2cm}

\begin{center}
  \begin{tabular}{ll}
    $p^{\text{\tmtextsc{EviiL}}} \assign K p$ &
    $\bot^{\text{\tmtextsc{EviiL}}} \assign \bot$\\
    $(\phi \vee \psi)^{\text{\tmtextsc{EviiL}}} \assign
    (\phi^{\text{\tmtextsc{EviiL}}} \vee \psi^{\text{\tmtextsc{EviiL}}})^{}$ &
    $(\phi \wedge \psi)^{\text{\tmtextsc{EviiL}}} \assign
    (\phi^{\text{\tmtextsc{EviiL}}} \wedge
    \psi^{\text{\tmtextsc{EviiL}}})^{}$\\
    $(\phi \rightarrow \psi)^{\text{\tmtextsc{EviiL}}} \assign \boxplus
    (\phi^{\text{\tmtextsc{EviiL}}} \rightarrow
    \psi^{\text{\tmtextsc{EviiL}}})$ & 
  \end{tabular}
\end{center}

Where $K \phi \assign \ominus (\circlearrowleft \wedge \Box  \phi)$, the
formulation of knowledge previously suggested

\framebreak

One again, we have:
\[ \vdash_{\tmop{Int}} \phi \Longleftrightarrow
   \vdash_{\text{\tmtextsc{EviL}}} \phi^{\text{\tmtextsc{EviiL}}} \]
Intuitionistic Logic as the Logic of Imagination

\vspace{.2cm}

One way to read $\Box  \phi$ is ``can deduce $\phi$'', and its dual
$\diamondsuit \phi$ as ``can imagine $\phi$''

\framebreak

Because of the symmetry of $\boxplus$ and $\boxminus$, and their interplay
with $\Box $ and $\diamondsuit$ as previously illustrated, we have yet another
embedding $(\cdot)^{\text{\tmtextsc{EviiiL}}} :
\text{$\mathcal{L}_{\tmop{Int}}$} \rightarrow
\mathcal{L}_{\text{\tmtextsc{EviL}}}$ of intuitionistic logic into
\tmtextsc{EviL}:

\begin{center}
  \begin{tabular}{l}
    $p^{\text{\tmtextsc{EviiiL}}} \assign \diamondsuit p$\\
    $\bot^{\text{\tmtextsc{EviiiL}}} \assign \bot$\\
    $(\phi \vee \psi)^{\text{\tmtextsc{EviiiL}}} \assign
    (\phi^{\text{\tmtextsc{EviiiL}}} \vee
    \psi^{\text{\tmtextsc{EviiiL}}})^{}$ \\
    $(\phi \wedge \psi)^{\text{\tmtextsc{EviiiL}}} \assign
    (\phi^{\text{\tmtextsc{EviiiL}}} \wedge
    \psi^{\text{\tmtextsc{EviiiL}}})^{}$\\
    $(\phi \rightarrow \psi)^{\text{\tmtextsc{EviiiL}}} \assign \boxminus
    (\phi^{\text{\tmtextsc{EviiiL}}} \rightarrow
    \psi^{\text{\tmtextsc{EviiiL}}})$ 
  \end{tabular}
\end{center}

As before, we have:
\[ \vdash_{\tmop{Int}} \phi \Longleftrightarrow
   \vdash_{\text{\tmtextsc{EviL}}} \phi^{\text{\tmtextsc{EviiiL}}} \]
\end{frame}

\begin{frame}[allowframebreaks]
\frametitle{Modal Intuitionistic Logic}

Furthermore, this embedding can be extended to the modal intuitionistic logic
$\tmop{ImK}_{\Box }$:

\vspace{.2cm}

{\hspace*{\fill}}$\phi \text{ } \colons = \text{ } p \text{ } | \text{ } \bot
\text{ } | \text{ } \phi \vee \psi \text{ } | \text{ } \phi \wedge \psi \text{
} | \text{ } \phi \rightarrow \psi \text{ } | \text{ } \Box 
\phi${\hspace*{\fill}}($\mathcal{L}_{\tmop{ImK}_{\Box }}$)

\framebreak

This gives rise to $(\cdot)^{\text{\tmtextsc{EixL}}} :
\text{$\mathcal{L}_{\tmop{ImK}_{\Box }}$} \rightarrow
\mathcal{L}_{\text{\tmtextsc{EviL}}}$

\begin{center}
  \begin{tabular}{ll}
    $p^{\text{\tmtextsc{EixL}}} \assign \Box p$ &
    $\bot^{\text{\tmtextsc{EixL}}} \assign \bot$\\
    $(\phi \vee \psi)^{\text{\tmtextsc{EixL}}} \assign
    (\phi^{\text{\tmtextsc{EixL}}} \vee \psi^{\text{\tmtextsc{EixL}}})^{}$ &
    $(\phi \wedge \psi)^{\text{\tmtextsc{EixL}}} \assign
    (\phi^{\text{\tmtextsc{EixL}}} \wedge \psi^{\text{\tmtextsc{EixL}}})^{}$\\
    $(\phi \rightarrow \psi)^{\text{\tmtextsc{EixL}}} \assign \boxplus
    (\phi^{\text{\tmtextsc{EixL}}} \rightarrow \psi^{\text{\tmtextsc{EixL}}})$
    & $(\Box  \phi)^{\text{\tmtextsc{EixL}}} \assign \Box 
    \phi^{\text{\tmtextsc{EixL}}}$
  \end{tabular}
\end{center}

\vspace{.2cm}

Not surprisingly, we have:
\[ \vdash_{\tmop{ImK}_{\Box }} \phi \Longleftrightarrow
   \vdash_{\text{\tmtextsc{EviL}}} \phi^{\text{\tmtextsc{EixL}}} \]
\end{frame}

\begin{frame}
\frametitle{Conclusion}

\tmtextsc{EviL} admittedly presents a somewhat different perspective on
knowledge than traditional epistemic logic

\vspace{.5cm}

Regardless, the tools of \tmtextsc{EviL} and \tmtextsf{EL} are the same; they
are after all both modal logics

\vspace{.5cm}

It is my sincerest hope that \tmtextsc{EviL} will help the field to progress
into new areas of investigation
\end{frame}
% \end{document}

%%% Local Variables: 
%%% mode: latex
%%% TeX-master: "beamer"
%%% End: 
